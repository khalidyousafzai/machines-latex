\باب{مقناطیسی ادوار}
%proof read urdu
\حصہ{مزاحمت  اور ہچکچاہٹ}
شکل \حوالہ{شکل_مقناطیسی_دور_مزاحمت_ہچکچاہٹ}  میں ایک سلاخ دکھائی گئی ہے جس کی لمبائی کے رخ   \اصطلاح{مزاحمت}\فرہنگ{مزاحمت}\حاشیہب{resistance}\فرہنگ{resistance}  
\begin{align}\label{مساوات_مقناطیسی_دور_مزاحمت_کی_تعریف}
R=\frac{l }{\sigma A}
\end{align}
ہو گی جہاں  \عددیء{\sigma} \اصطلاح{موصلیت}\فرہنگ{موصلیت}\حاشیہب{conductivity}\فرہنگ{conductivity}  اور \عددیء{A=wh} رقبہ عمودی تراش   ہے۔
\begin{figure}
\centering
\begin{tikzpicture}
\pgfmathsetmacro{\height}{0.75}  
\pgfmathsetmacro{\widthX}{0.25}  
\pgfmathsetmacro{\widthY}{0.25}  
\pgfmathsetmacro{\length}{1.75}  
\coordinate(O) at (0,0,0);
%rod 3d
\draw[black] (O)--(\length,0)node[pos=0.5,below]{$l$}--++(0,\height)--++(-\length,0)--cycle;
\draw[black] (O)++(\length,0)--++(\widthX,\widthY)node[pos=0.4,right]{$w$}--++(0,\height)node[pos=0.5,right]{$h$}--++(-\length,0)--++(-\widthX,-\widthY);
\draw[black] (O)++(\length,0)++(0,\height)--++(\widthX,\widthY);
\draw[gray,-latex] (O)++(0.3*\length,0.5*\height)--++(0.4*\length,0);
\coordinate(textarea) at (\length,2*\height);
\coordinate(flowMid) at (0.5*\length,0.5*\height);
\coordinate(mathEq) at (3*\length,0.7*\height);
\draw[gray,thin,dashed,<-] (flowMid) to [out=90,in=270] (textarea);
%text
\node[anchor = south] at (textarea) {\text{\RL{برقی رو یا مقناطیسی بہاو کا رخ}}};
\node[anchor = east] at (mathEq) {$
\begin{aligned}
R&=\frac{l}{\sigma A}\\
\Re &=\frac{l}{\mu A}
\end{aligned}
$};
\end{tikzpicture}%
\caption{مزاحمت اور ہچکچاہٹ}
\label{شکل_مقناطیسی_دور_مزاحمت_ہچکچاہٹ}
\end{figure}
اس سلاخ کی \اصطلاح{ہچکچاہٹ}\فرہنگ{ہچکچاہٹ}\فرہنگ{reluctance}\حاشیہب{reluctance} \عددیء{\Re}  درج ذیل ہے جہاں \عددیء{\mu}  \اصطلاح{مقناطیسی مستقل}\فرہنگ{مقناطیسی مستقل}\حاشیہب{permeability, magnetic constant}\فرہنگ{permeability}\فرہنگ{magnetic constant} کہلاتا ہے۔ 
\begin{align}\label{مساوات_مقناطیسی_دور_ہچکچاہٹ_کی_تعریف}
\Re = \frac{l}{\mu A}
\end{align}
مقناطیسی مستقل \عددیء{\mu} کو عموماً خلاء کی مقناطیسی مستقل \عددیء{\mu_0=4\pi\, 10^{-7}\,\si{\henry\per\meter}} کی نسبت سے لکھا جاتا ہے یعنی
\begin{align}
\mu=\mu_r \mu_0
\end{align}
جہاں \عددیء{\mu_r} \اصطلاح{جزو مقناطیسی مستقل}\فرہنگ{مقناطیسی مستقل!جزو}\فرہنگ{relative permeability}\فرہنگ{permeability!relative}  کہلاتا ہے۔ہچکچاہٹ کی اکائی \اصطلاح{ایمپیئر و چکر فی ویبر}  ہے جس کی وضاحت جلد کی جائے گی۔
%
\ابتدا{مثال}
شکل \حوالہ{شکل_مقناطیسی_دور_مزاحمت_ہچکچاہٹ}  میں دی گئی سلاخ کی ہچکچاہٹ معلوم کریں جہاں 
\عددیء{\mu_r=2000 }، \عددیء{l=\SI{10}{\centi \meter}}، \عددیء{h=\SI{3}{\centi \meter}} اور \عددیء{w=\SI{2.5}{\centi \meter}} ہیں۔

حل:
\begin{align*}
\Re& = \frac{l}{\mu_r \mu_0 A}\\
&=\frac{10\times 10^{-2}}{2000 \times 4 \pi \times 10^{-7} \times 2.5 \times 10^{-2} \times 3 \times 10^{-2}}\\
&=\SI{53052}{\ampere \cdot turns \per \weber}
\end{align*}
\انتہا{مثال}

\حصہ{کثافتِ برقی رو  اور برقی میدان کی شدت}\شناخت{حصہ_برقی_دور_کثافت_برقی_رو_اور_میدان}
شکل  \حوالہ{شکل_مقناطیسی_دور_کثافت_رو_اور_برقی_شدت} میں ایک موصل سلاخ کے سروں پر برقی دباو \عددی{v}   لاگو کیا گیا ہے۔سلاخ میں  برقی رو \عددیء{i}  \اصطلاح{اوہم} کے قانون\فرہنگ{قانون!اوہم}\حاشیہب{Ohm's law}\فرہنگ{Ohm's law}  سے حاصل ہو گا۔
\begin{align}
i=\frac{v}{R}
\end{align}
درج بالا مساوات کو مساوات \حوالہ{مساوات_مقناطیسی_دور_مزاحمت_کی_تعریف}  کی مدد سے
\begin{align}
i=v \left(\frac{\sigma A}{l}\right)
\end{align}
یعنی
\begin{align}
\frac{i}{A}=\sigma \left(\frac{v}{l} \right)
\end{align}
یا
\begin{align}\label{مساوات_مقناطیسی_دور_اوہم_قانون_کی_تفرق_شکل}
J =\sigma E
\end{align}
لکھا جا سکتا ہے جہاں  \عددی{J} اور \عددی{E} کی تعریفات درج ذیل ہیں۔
\begin{align}
J&=\frac{i}{A} \label{مساوات_مقناطیسی_دور_کثافت_رو}\\
E&=\frac{v}{l} \label{مساوات_مقناطیسی_دور_برقی_شدت}
\end{align}

شکل  \حوالہ{شکل_مقناطیسی_دور_کثافت_رو_اور_برقی_شدت} میں سمتیہ \سمتیہ{J} کی مطلق قیمت \عددیء{J}  اور سمتیہ \سمتیہ{E} کی مطلق قیمت \عددی{E} لیتے ہوئے  مساوات \حوالہ{مساوات_مقناطیسی_دور_اوہم_قانون_کی_تفرق_شکل} کو درج ذیل لکھا جا سکتا ہے
\begin{align}
\kvec{J}=\sigma \kvec{E}
\end{align}
جو قانون اوہم کی دوسری روپ ہے۔ \سمتیہ{J} اور \سمتیہ{E} دونوں کا رخ  \عددیء{\ay}  ہے۔

%
\begin{figure}
\centering
\begin{tikzpicture}
 \pgfmathsetmacro{\height}{0.75}  
\pgfmathsetmacro{\widthX}{0.25}  
\pgfmathsetmacro{\widthY}{0.25}  
\pgfmathsetmacro{\length}{1.75}  
%\draw[gray,thick] (0,0) grid (2,2);
%\draw[help lines,xstep=0.1,ystep=0.1] (0,0) grid (2,2);
\coordinate(O) at (0,0,0);

%rod 3d
\draw[black] (O)--(\length,0)node[pos=0.3,below]{$l$}--++(0,\height)--++(-\length,0)--cycle;
\draw[black] (O)++(\length,0)--++(\widthX,\widthY)--++(0,\height)--++(-\length,0)--++(-\widthX,-\widthY);
\draw[black] (O)++(\length,0)++(0,\height)--++(\widthX,\widthY);
%cross section
\draw[gray,dashed](O)++(0.4*\length,0)--++(0,\height)--++(\widthX,\widthY)--++(0,-\height)--cycle;
%current arrow
\draw[gray,-latex] (O)++(0.4*\length,0.5*\height)++(0.5*\widthX,0.5*\widthY)--++(0.3*\length,0) node [below] {$i$};
%circuit
\draw (O)++(0,0.5*\height+0.12)--++(-0.2,0)--++(0,1.75)--++(0.5*\length+0.2,0);  %left section
\draw[thick] (O)++(0,0.5*\height+0.12)++(-0.2,0)++(0,1.75)++(0.5*\length+0.2,0)++(0,0.2)--++(0,-0.4)node[below]{$v$}; 
\draw (O)++(\length,0.5*\height)++(0.5*\widthX,0.5*\widthY)--++(0.2,0)--++(0,1.75)--++(-0.5*\length-0.2,0)++(0,0.1)--++(0,-0.2); %right section

\draw[gray] (0.85,0.155) to [out=-45,in=180] (1.5,-0.5) node[right] {$A$};
%unit vector
\draw[-latex] (O)++(-1.8,0)--++(-1.4142*\widthX,-1.4142*\widthY)node[below]{${\bf{a}}_x$};
\draw[-latex] (O)++(-1.8,0)--++(2*\widthX,0)node[right]{${\bf{a}}_y$};
\draw[-latex] (O)++(-1.8,0)--++(0,2*\widthY)node[above]{${\bf{a}}_z$};
%equations
\draw (O)++(2*\length,-0.6) node [above right]{$
\begin{aligned}
R&=\frac{l}{\sigma A}\\
i&=\frac{v}{R}=v \left( \frac{\sigma A}{l} \right)\\
\frac{i}{A}&=\sigma \frac{v}{l}\\
J&=\sigma E
\end{aligned}
$};
\end{tikzpicture}%
\caption{کثافتِ برقی رو اور برقی دباو کی شدت}
\label{شکل_مقناطیسی_دور_کثافت_رو_اور_برقی_شدت}
\end{figure}
شکل \حوالہ{شکل_مقناطیسی_دور_کثافت_رو_اور_برقی_شدت} سے ظاہر ہے کہ برقی رو \عددیء{i} سلاخ کے رقبہ عمودی تراش \عددیء{A} سے گزرتا ہے لہٰذا مساوات \حوالہ{مساوات_مقناطیسی_دور_کثافت_رو} کے تحت \عددیء{J}، \اصطلاح{کثافتِ برقی رو}\فرہنگ{کثافت!برقی رو}\حاشیہب{current density} ہو گا۔ اسی طرح مساوات \حوالہ{مساوات_مقناطیسی_دور_برقی_شدت}   سے  واضح ہے کہ \عددیء{E} برقی دباو فی اکائی لمبائی کو ظاہر کرتی ہے لہٰذا  \عددیء{E} کو \اصطلاح{برقی میدان کی شدت}\فرہنگ{برقی میدان!شدت}\حاشیہب{electric field intensity} یا (جہاں متن سے مقناطیسی میدان واضح ہو) مختصراً \اصطلاح{میدانی شدت}  کہتے ہیں۔

بالکل اسی طرح کی مساواتیں مقناطیسی متغیرات کے لئے حصہ \حوالہ{حصہ_برقی_دور_کثافت_مقناطیسی_بہاو_اور_میدان}  میں لکھی جائیں گی۔  

\حصہ{برقی ادوار}
برقی دور میں \اصطلاح{برقی دباو}\فرہنگ{برقی دباو}\حاشیہب{electric voltage}  \عددیء{v}\حاشیہد{برقی دباو کی اکائی وولٹ ہے جو اٹلی کے الِسانڈرو وولٹا کے نام ہے جنہوں نے برقی بیٹری ایجاد کی۔}  کی وجہ سے \اصطلاح{برقی رو}\فرہنگ{برقی رو}\حاشیہب{electric current} \عددیء{i} \حاشیہد{برقی رو کی اکائی ایمپیئر ہے جو فرانس کے انڈرِ میرِ ایمپیئر کے نام ہے جن کا برقی و مقناطیسی میدان میں اہم کردار ہے۔} پیدا ہوتا ہے۔ تانبا\فرہنگ{تانبا}\حاشیہب{copper}   کی موصلیت \عددیء{\sigma=\SI{5.9e7}{\siemens \per \meter}} ہے جو بہت بڑی مقدار ہے۔ موصلیت کی اکائی \عددیء{\si{\siemens \per \meter}}  ہے۔ تانبا کی موصلیت کی مقدار بہت بڑی ہونے کی بنا اس سے بنی تار کی مزاحمت\حاشیہد{مزاحمت کی اکائی اوہم ہے جو جرمنی کے جارج سائمن اوہم کے نام ہے جنہوں نے قانونِ اوہم دریافت کیا۔}  \عددیء{R_{\textup{تار}}}  عموماً قابلِ نظرانداز ہو گی۔  تار میں برقی رو \عددیء{i} گزرنے سے تار کے سروں کے بیچ برقی دباو  \عددیء{\Delta v=i R_{\textup{تار}}} پیدا ہو گا جس کو  \عددیء{R_{\textup{تار}}\to 0} کی بنا  نظر انداز کیا جا سکتا ہے۔یوں تانبے کی تار میں برقی دباو کے گھٹاو کو رد کیا جا سکتا ہے یعنی ہم \عددیء{\Delta v \to 0} لے سکتے ہیں۔ 

شکل \حوالہ{شکل_مقناطیسی_دور_سلسہ_وار_مزاحمتی_ادوار}-الف میں ایک ایسا ہی برقی دور دکھایا گیا ہے جس  میں تانبے کی تار کی مزاحمت کو اکٹھے کر کے ایک ہی جگہ  \عددیء{R_{\textup{تار}}} دکھایا گیا ہے۔اس دور کے لئے درج ذیل لکھا جا سکتا ہے۔
\begin{align}
v&=\Delta v+v_L
\end{align}
تار میں برقی گھٹاو \عددیء{\Delta v} نظرانداز کرتے ہوئے
\begin{align}
v&=v_L
\end{align}
حاصل ہوتا ہے۔اس کا مطلب ہوا کہ  تار میں برقی دباو کا گھٹاو قابل نظرانداز ہونے کی صورت میں لاگو برقی دباو جوں کا توں مزاحمت \عددیء{R_L} تک پہنچتا ہے۔برقی ادوار حل کرتے ہوئے یہی حقیقت بروئے کار لاتے ہوئے تار میں برقی دباو کے گھٹاو کو نظرانداز کیا جاتا ہے۔شکل \حوالہ{شکل_مقناطیسی_دور_سلسہ_وار_مزاحمتی_ادوار}-الف میں ایسا کرنے سے  شکل \حوالہ{شکل_مقناطیسی_دور_سلسہ_وار_مزاحمتی_ادوار}-ب حاصل ہوتا ہے۔یہاں یہ سمجھ لینا ضروری ہے کہ برقی تار کو اس غرض سے استعمال کیا جاتا ہے کہ لاگو برقی دباو کو مقام استعمال تک بغیر گھٹائے پہنچایا جائے۔
\begin{figure}
\centering
\begin{subfigure}[b]{0.45\textwidth}
\centering
\begin{tikzpicture}[american voltages]
\draw (0,0)--(1.5*\xx,0) to [european resistor,l={$R_L$},a={$\begin{aligned}&+\\  &v_L\\  &- \end{aligned}$}] ++(0,\xx) to
 [resistor,a={$+\,\Delta v\,-$},l={$R_{\textup{تار}}$}]++(-\xx,0) to [short,i<=$i$]++(-0.5*\xx,0) to [battery,l_={$v$}] (0,0);
\end{tikzpicture}%
\caption{}
\end{subfigure}\hfill
\begin{subfigure}[b]{0.45\textwidth}
\centering
\begin{tikzpicture}[american voltages]
\draw (0,0)--(\xx,0) to [european resistor,l_={$R_L$}] ++(0,\xx) to [short,i<=$i$] ++(-\xx,0) to [battery,l_={$v$}] (0,0); 
%equations
\draw[above,right] (1.5*\xx,\yy/2) node[right]{$
\begin{aligned}
\Delta v&=i R_{\textup{تار}}\\
R_{\textup{تار}}& \to 0\\
\Delta v& \to 0
\end{aligned}
$};
\end{tikzpicture}%
\caption{}
\end{subfigure}%
\caption{برقی ادوار میں برقی تار کی مزاحمت کو نظر انداز کیا جا سکتا ہے۔}
\label{شکل_مقناطیسی_دور_سلسہ_وار_مزاحمتی_ادوار}
\end{figure}%
%---------------------
\begin{figure}
\centering
\begin{subfigure}[b]{0.40\textwidth}
\centering
\begin{tikzpicture}[american voltages]
\draw (0,0)--(1.5*\xx,0) to [european resistor,l_={$R_2$},i<_=$i_2$]++(0,\yy) to [short,i<=$i_t$]++(-1.5*\xx,0) to [battery,l_={$v$}] (0,0); 
\draw (\xx,\yy) to [european resistor,l_={$R_1$},i_=$i_1$ ,*-*]++(0,-\yy);
\end{tikzpicture}%
\caption{}
\end{subfigure}\hfill
\begin{subfigure}[b]{0.50\textwidth}
\centering
\begin{tikzpicture}[american voltages]
\draw (0,0)--(\xx,0) to [european resistor,l_={$R_t$}] ++(0,\yy) to  [short,i<=$i_t$]++(-\xx,0) to [battery,l_={$v$}] (0,0); 
%equations
\draw (1.5*\xx,\yy/2) node[above,right]{$
\begin{aligned}
\frac{1}{R_t}&=\frac{1}{R_1}+\frac{1}{R_2}\\
i_1&=\frac{v}{R_1}\\
i_2&=\frac{v}{R_2}
\end{aligned}
$};
\end{tikzpicture}%
\caption{}
\end{subfigure}%
\caption{کم مزاحمتی راہ میں برقی رو کی مقدار زیادہ ہو گی۔}
\label{شکل_مقناطیسی_دور_متوازی_مزاحمتی_دور}
\end{figure}%
%

شکل \حوالہ{شکل_مقناطیسی_دور_متوازی_مزاحمتی_دور}  میں دوسری مثال دی گئی ہے۔ یہاں ہم دیکھتے ہیں کہ برقی رو اس راہ زیادہ ہو گا جس کی مزاحمت کم ہو۔ یوں  \عددیء{R_1 < R_2} کی صورت میں \عددیء{i_1>i_2} ہو گا۔


\حصہ{مقناطیسی دور حصہ اول}
مقناطیسی ادوار بالکل برقی ادوار کی طرح ہوتے ہیں۔ بس ان میں برقی دباو \عددیء{v} کی جگہ \اصطلاح{مقناطیسی دباو}\فرہنگ{مقناطیسی دباو}\حاشیہب{magnetomotive force, mmf}\فرہنگ{mmf} \عددیء{\tau} ، برقی رو \عددیء{i}  کی جگہ \اصطلاح{مقناطیسی بہاو}\فرہنگ{مقناطیسی بہاو}\حاشیہب{flux}\فرہنگ{flux} \عددیء{\phi}  اور مزاحمت \عددیء{R} کی جگہ  \اصطلاح{ہچکچاہٹ}\فرہنگ{ہچکچاہٹ}\حاشیہب{reluctance}  \عددیء{\Re} پائے جاتے ہیں۔ یوں  بالکل برقی ادوار کی طرح مقناطیسی ادوار بنائے جا سکتے ہیں۔  ایسا ایک مقناطیسی دور شکل \حوالہ{شکل_مقناطیسی__مقناطیسی_سلسلہ_وار_دور}-الف میں دکھایا گیا ہے۔
\begin{figure}
\centering
\begin{subfigure}[b]{0.40\textwidth}
\centering
\begin{tikzpicture}[american voltages]
\draw (0,0)--(1.5*\xx,0) to [european resistor,l_={$\Re_a$}]++(0,\yy) to [european resistor,l={$\Re_c$},a={$+\, \Delta \tau\, -$}]++ (-\xx,0) to [short,i<=$\phi$]++(-0.5*\xx,0) to [battery,l_={$\tau$}] (0,0); 
\end{tikzpicture}%
\caption{}
\end{subfigure}\hfill
\begin{subfigure}[b]{0.50\textwidth}
\centering
\begin{tikzpicture}[american voltages]
\draw (0,0)--(\xx,0) to [european resistor,l_={$\Re_a$}]++(0,\yy) to [short,i<=$\phi$]++(-\xx,0) to [battery,l_={$\tau$}] (0,0); 
%equations
\draw[] (1.5*\xx,\yy/2) node[right]{$
\begin{aligned}
\Delta \tau&=\phi \Re_c\\
\Re_c& \to 0\\
\Delta \tau& \to 0
\end{aligned}
$};
\end{tikzpicture}
\caption{}
\end{subfigure}
\caption{مقناطیسی دور}
\label{شکل_مقناطیسی__مقناطیسی_سلسلہ_وار_دور}
\end{figure}
%
یہاں بھی کوشش یہی ہے کہ  مقناطیسی دباو \عددیء{\tau}  بغیر گھٹائے ہچکچاہٹ \عددیء{\Re_a} تک پہنچایا جائے۔ خلائی درز کی ہچکچاہٹ  \عددیء{\Re_a}   اور مقناطیسی راہ کی ہچکچاہٹ \عددیء{\Re_c}  ہے۔ یوں \عددیء{\Re_c} قابل نظرانداز ہونے کی صورت میں  شکل \حوالہ{شکل_مقناطیسی__مقناطیسی_سلسلہ_وار_دور}-ب حاصل ہو گا جس میں مقناطیسی بہاو \عددیء{\phi}، بالکل اوہم کے قانون کی طرح، درج ذیل مساوات سے حاصل ہو گا۔
\begin{align}\label{مساوات_مقناطیسی_دور_قانون_اوہم}
\tau=\phi \Re_a
\end{align}
جہاں  \عددیء{\Re_c} قابل  نظرانداز ہو وہاں، سلسلہ وار مزاحمتوں کی طرح،  دو سلسلہ وار ہچکچاہٹوں کا مجموعی ہچکچاہٹ  \عددیء{\Re_s}  استعمال کر کے برقی بہاو حاصل ہو گا۔
\begin{align}
\Re_s&=\Re_a+\Re_c\\
\tau&=\phi \Re_s \label{مساوات_مقناطیسی_دور_مقناطیسی_اوہم_قانون}
\end{align}

برقی دور کی طرح، مقناطیسی دباو کو کم ہچکچاہٹ کی راہ استعمال کرتے ہوئے مقام ضرورت تک پہنچایا جاتا ہے۔ مساوات \حوالہ{مساوات_مقناطیسی_دور_ہچکچاہٹ_کی_تعریف}   کے تحت  ہچکچاہٹ کی قیمت  مقناطیسی مستقل \عددیء{\mu} پر منحصر ہے ۔مقناطیسی مستقل کی اکائی   ہینری فی میٹر \عددیء{\si{\henry \per \meter}} ہے۔\عددیء{\mu} کو عموماً \عددیء{\mu=\mu_r \mu_0} لکھا جاتا ہے جہاں  \عددیء{\mu_0=4 \pi \times 10^{-7}} ہینری فی میٹر کے برابر ہے اور \عددیء{\mu_r} کو \اصطلاح{جزو مقناطیسی مستقل}\فرہنگ{مقناطیسی مستقل!جزو}\حاشیہب{relative permeability, relative magnetic constant} کہتے ہیں۔ لوہا،  کچھ دھاتیں اور چند جدید مصنوعی مواد  ایسی ہیں جن کی \عددیء{\mu_r} کی قیمت \عددیء{\num{2000}} اور \عددیء{\num{80000}} کے  بیچ پائی جاتی ہیں۔ مقناطیسی دباو کو  ایک مقام سے دوسری مقام منتقل کرنے کے لئے ان ہی مقناطیسی مواد کو  استعمال کیا جاتا ہے۔

 بد قسمتی سے  مقناطیسی مواد کے  \عددیء{\mu} کی قیمت اتنی زیادہ  نہیں ہوتی ہے کہ ان سے بنی سلاخ کی ہچکچاہٹ ہر موقع پر قابل نظرانداز ہو۔ مساوات \حوالہ{مساوات_مقناطیسی_دور_ہچکچاہٹ_کی_تعریف}  کے تحت  ہچکچاہٹ کم سے کم کرنے کی خاطر رقبہ عمودی تراش کو زیادہ سے زیادہ اور لمبائی کو کم سے کم  کرنا ہو گا۔ یوں مقناطیسی دباو منتقل کرنے کے لئے  باریک تار نہیں بلکہ خاصا زیادہ رقبہ عمودی تراش کا مقناطیسی راستہ  درکار ہوتا ہے۔

مقناطیسی مشین، مثلاً موٹر اور ٹرانسفارمر، کا بیشتر حصہ مقناطیسی دباو منتقل کرنے والے ان مقناطیسی مواد  پر مشتمل ہوتا ہے۔ایسے مشینوں کے قلب میں عموماً یہی مقناطیسی مادہ پایا جاتا ہے لہٰذا ایسا مواد  \اصطلاح{مقناطیسی قالب}\فرہنگ{مقناطیسی قالب}\حاشیہب{magnetic core}\فرہنگ{magnetic core} کہلاتا ہے (شکل \حوالہ{شکل_مقناطیسی__کثافت_مقناطیسی_بہاو_اور_شدت})۔

برقی مشینوں میں مستعمل  مقناطیسی قالب لوہے کی باریک چادر یا پتری\فرہنگ{پتری}\حاشیہب{laminations}\فرہنگ{laminations}  تہہ  در تہہ رکھ کر بنایا جاتا ہے۔ مقناطیسی قالب کے بارے میں مزید معلومات حصہ \حوالہ{حصہ_مقناطیسی_دور_مقناطیسی_مادہ_کے_خصوصیات}  میں  فراہم کی جائے گی۔

\حصہ{کثافتِ مقناطیسی بہاو  اور مقناطیسی میدان کی شدت}\شناخت{حصہ_برقی_دور_کثافت_مقناطیسی_بہاو_اور_میدان}
حصہ \حوالہ{حصہ_برقی_دور_کثافت_برقی_رو_اور_میدان}  میں  برقی دور کی مثال دی گئی۔یہاں شکل \حوالہ{شکل_مقناطیسی__کثافت_مقناطیسی_بہاو_اور_شدت} میں دکھائے گئے مقناطیسی دور پر غور کرتے ہیں۔  مقناطیسی قالب کا \عددیء{\mu_r = \infty} تصور کرتے ہوئے آگے بڑھتے ہیں۔ یوں  قالب کی ہچکچاہٹ \عددیء{\Re_c} صفر ہو گی۔ حصہ \حوالہ{حصہ_برقی_دور_کثافت_برقی_رو_اور_میدان}   میں تانبا کی تار کی طرح یہاں  مقناطیسی قالب کو مقناطیسی دباو \عددیء{\tau} ایک مقام سے دوسری مقام تک منتقل کرنے کے لئے استعمال کیا گیا ہے۔ شکل \حوالہ{شکل_مقناطیسی__کثافت_مقناطیسی_بہاو_اور_شدت} میں مقناطیسی دباو کو خلائی درز کی ہچکچاہٹ \عددیء{\Re_a} تک پہنچایا گیا ہے۔ یہاں \عددیء{\Re_c} کو نظرانداز کرتے ہوئے  کل ہچکچاہٹ کو خلائی درز کی ہچکچاہٹ کے برابر تصور کیا جا سکتا ہے:
\begin{figure}
\centering
\begin{tikzpicture}
\def\height{2};
\def\width{1.5};
\def\thick{0.4};
\def\depthX{0.2};
\def\depthY{0.2};
\def\gap{0.05};
\pgfmathsetmacro{\widthX}{0.25}  
\pgfmathsetmacro{\widthY}{0.25}  
\coordinate(O) at (-0.5,0);
%grid
%\draw[gray,thick](0,0) grid (12,3);
%\draw[gray,thin,xstep=0.1,ystep=0.1](0,0) grid (12,3);
%going clockwise from origin
\draw(0,0)--++(0,\height)--++(\width,0)node[pos=0.5,pin=45:قالب{}]{}--++(0,-0.5*\height+\gap)--++(-\thick,0)--++(0,0.5*\height-\gap-\thick)--++(-\width+2*\thick,0)--++(0,-\height+2*\thick)--++(\width-2*\thick,0)--++(0,0.5*\height-\thick-\gap)--++(\thick,0)--++(0,-0.5*\height+\gap)--cycle;
%
\draw(\thick,\thick)--++(\depthX,\depthY) --++(0,\height-2*\thick-\depthY);
\draw(0.4,0.4)--++(\width-2*\thick-\depthX,0);
\draw(0,\height)--++(\depthX,\depthY)--++(\width,0)--++(-\depthX,-\depthY);
\draw(\width,\height)++(\depthX,\depthY)--++(0,-0.5*\height+\gap)--++(-\depthX,-\depthY);
\draw(\width,0)--++(\depthX,\depthY)--++(0,0.5*\height-\gap)--++(-\depthX,-\depthY);
\draw(\thick+\depthX,\thick+\depthY)--++(\width-2*\thick-\depthX,0);
%gap
\draw(1.7,1.155)--++(-0.1,0);
\draw(1.1,0.95)--++(0.1,0.1);
%winding
\draw (0.6,1.4) to [out=45,in=0] (0.2,1.5) to [short,i_<=$i$] (-1,1.5) node[left]{$+$};
\foreach \l in {1.4,1.2,1}{
\draw (0,\l) to [out=-135,in=45] (0.6,\l-0.2);
}
\draw (0,0.8) to (-1,0.8)node[left]{$-$};
%turns
\node at (0,1.15)[left]{$\tau=N i$};
%gap
\draw[gray](1.8,1.15)--++(0.3,0);
\draw[gray](1.8,1.25)--++(0.3,0);
\draw[->](2.2,1.7)node[right]{$l_a$}--++(-0.2,-0.2)--++(0,-0.25);
\draw[->](2,1)--++(0,0.15);
%flux
\draw[-latex,gray](1.1,1.8)node[left,black]{$\phi$}--++(0.2,0)--++(0,-\height+\thick)--++(-\width+\thick,0)--++(0,\height-\thick)--++(0.5,0);
%urdu coil
\draw[thin,<-](0.3,0.9) to [out=-90,in=60] (-0.7,-0.3)node[below]{\RL{چکر کا لچھا}$\,N$};
%equations
\node at (4,1)[right]{$
\begin{aligned}
H_a&=\frac{\tau}{l_a}\quad \quad B_a&=\frac{\phi_a}{A_a}\\
l_a& \ll w \\
l_a&\ll b 
\end{aligned}
$};
%dimensions
\draw[stealth-stealth] (1.1,-0.1)--++(0.4,0)node[below,pos=0.5]{$b$};
\draw[stealth-stealth](1.5+0.1,-0.1)--++(0.2,0.2)node [pos=0.4,right]{$w$};
%unit vector
\draw[-latex] (O)++(-1.8,0)--++(-1.4142*\widthX,-1.4142*\widthY)node[below]{${\bf{a}}_x$};
\draw[-latex] (O)++(-1.8,0)--++(2*\widthX,0)node[right]{${\bf{a}}_y$};
\draw[-latex] (O)++(-1.8,0)--++(0,2*\widthY)node[above]{${\bf{a}}_z$};
\end{tikzpicture}%
\caption{کثافتِ مقناطیسی بہاو اور مقناطیسی میدان کی شدت۔}
\label{شکل_مقناطیسی__کثافت_مقناطیسی_بہاو_اور_شدت}
\end{figure}
%
\begin{align}
\Re_a=\frac{l_a}{\mu_0 A_a}
\end{align}
خلائی درز کی لمبائی \عددیء{l_a} قالب کے رقبہ عمودی تراش کے اضلاع \عددیء{b} اور \عددیء{w} سے بہت کم ہونے کی صورت ، یعنی  \عددیء{l_a \ll b} اور \عددیء{l_a \ll w}، میں خلائی درز کے رقبہ عمودی تراش \عددیء{A_a} کو قالب کے رقبہ عمودی تراش \عددیء{A_c} کے برابر تصور کیا  جا سکتا ہے:
\begin{align}
A_a=A_c=w b
\end{align}
اس کتاب میں جہاں بتلایا نہ گیا ہو وہاں \عددیء{l_a \ll b} اور \عددیء{l_a \ll w} تصور کرتے ہوئے \عددیء{A_a=A_c} لیا جائے گا۔
 
مقناطیسی دباو \عددی{\tau} کی تعریف درج ذیل مساوات  پیش کرتی ہے۔
\begin{align}
\tau=N i
\end{align}
یوں برقی تار کے چکر ضرب تار میں برقی رو کو مقناطیسی دباو کہتے ہیں۔ مقناطیسی دباو کی اکائی \اصطلاح{ایمپیئر و چکر}\فرہنگ{ایمپیئر و چکر}\حاشیہب{ampere-turn}\فرہنگ{ampere-turn}  ہے۔ حصہ \حوالہ{حصہ_برقی_دور_کثافت_برقی_رو_اور_میدان}   کی طرح ہم مساوات \حوالہ{مساوات_مقناطیسی_دور_مقناطیسی_اوہم_قانون} کو یوں لکھ سکتے ہیں۔
\begin{align}\label{مساوات_مقناطیسی_ڈور_بہاو_مساوی_دباو_بٹا_ہچکچاہٹ}
\phi_a=\frac{\tau}{\Re_a}
\end{align}
مقناطیسی بہاو کی اکائی\فرہنگ{ویبر}\حاشیہب{Weber}\فرہنگ{Weber} \اصطلاح{ویبر}\حاشیہد{یہ اکائی جرمنی کے ولیم اڈورڈ ویبر کے نام ہے جن کا برقی و مقناطیسی میدان میں اہم کردار رہا ہے}  اور ہچکچاہٹ کی اکائی \اصطلاح{ایمپیئر و چکر فی ویبر}\حاشیہب{ampere-turn per weber} ہے۔ اس سلسلہ وار دور کے خلائی درز میں مقناطیسی بہاو \عددیء{\phi_a} اور قالب میں مقناطیسی بہاو \عددیء{\phi_c} ایک دوسرے کے برابر ہوں گے۔درج بالا مساوات کو مساوات \حوالہ{مساوات_مقناطیسی_دور_ہچکچاہٹ_کی_تعریف}   کی مدد سے
\begin{align}
\phi_a &=\tau \left(\frac{\mu_0 A_a}{l_a} \right) \nonumber \\
\intertext{یا}
\frac{\phi_a}{A_a}&=\mu_0 \left( \frac{\tau}{l_a} \right) \label{مساوات_مقناطیسی_دور_کثافت_بہاو_اوہم_قانون_سے}
\end{align}
 لکھ سکتے ہیں جہاں درز کی نشاندہی زیر نوشت میں \عددی{a} لکھ کر کی گئی ہے۔ اس مساوات میں بائیں ہاتھ مقناطیسی بہاو فی اکائی رقبہ کو \اصطلاح{کثافتِ مقناطیسی بہاو}\فرہنگ{مقناطیسی بہاو!کثافت}\حاشیہب{magnetic flux density}\فرہنگ{magnetic flux!density} \عددیء{B_a} اور دائیں ہاتھ مقناطیسی دباو فی اکائی لمبائی کو \اصطلاح{مقناطیسی میدان کی شدت}\فرہنگ{مقناطیسی میدان!شدت}\حاشیہب{magnetic field intensity}\فرہنگ{magnetic field!intensity}  \عددیء{H_a} لکھا جا سکتا ہے:
\begin{align}
B_a&=\frac{\phi_a}{A_a}\\
H_a&=\frac{\tau}{l_a}
\end{align}
کثافتِ مقناطیسی بہاو کی اکائی \اصطلاح{ویبر فی مربع میٹر} ہے جس کو \اصطلاح{ٹسلا}\فرہنگ{ٹسلا}\فرہنگ{Tesla}\حاشیہد{Tesla:  یہ اکائی سربیا کے نِکولا ٹسلا کے نام ہے جنہوں نے بدلتا رو برقی طاقت عام کرنے میں اہم کردار ادا کیا۔}  کا نام دیا گیا ہے۔مقناطیسی میدان کی شدت کی اکائی \اصطلاح{ایمپیئر فی میٹر}\حاشیہب{ampere per meter}  ہے۔ یوں مساوات \حوالہ{مساوات_مقناطیسی_دور_کثافت_بہاو_اوہم_قانون_سے} کو درج ذیل لکھا جا سکتا ہے۔
\begin{align}
B_a=\mu_0 H_a
\end{align}
جہاں متن سے واضح ہو کہ مقناطیسی میدان کی بات ہو رہی ہے وہاں مقناطیسی میدان کی شدت کو مختصراً \اصطلاح{میدانی شدت}\حاشیہب{field intensity} کہا جاتا ہے۔

شکل \حوالہ{شکل_مقناطیسی__کثافت_مقناطیسی_بہاو_اور_شدت} میں خلائی درز میں مقناطیسی بہاو کا رخ  اکائی سمتیہ \عددیء{\az}  کا مخالف ہے لہٰذا کثافتِ مقناطیسی بہاو \عددیء{\kvec{B_a}=-B_a \az} لکھا جا سکتا ہے۔ اسی طرح خلائی درز میں مقناطیسی دباو  اکائی سمتیہ \عددیء{\az} کی مخالف رخ دباو ڈال رہا ہے لہٰذا مقناطیسی دباو کی شدت  \عددیء{\kvec{H_a}=-H_a \az} لکھی جائے گی۔ اس طرح درج بالا مساوات کو درج ذیل سمتی روپ میں  لکھا جا سکتا ہے۔
\begin{align}
\kvec{B_a}=\mu_0 \kvec{H_a}
\end{align}
خلاء کی جگہ کوئی دوسرا مادہ ہونے کی صورت میں یہ مساوات  درج ذیل روپ اختیار کرتی ہے۔
\begin{align}
\kvec{B}=\mu \kvec{H}
\end{align}
%
\ابتدا{مثال}
شکل \حوالہ{شکل_مقناطیسی__کثافت_مقناطیسی_بہاو_اور_شدت} میں خلائی درز میں کثافتِ مقناطیسی بہاو \عددیء{0.1} ٹسلا درکار ہے۔قالب کی \عددیء{\mu_r=\infty}  ہے، خلائی درز کی لمبائی \عددیء{1} ملی میٹر اور  قالب کے گرد برقی تار کے چکر  \عددیء{100} ہیں۔ درکار برقی رو \عددی{i} تلاش کریں۔

حل:\quad
مساوات \حوالہ{مساوات_مقناطیسی_دور_قانون_اوہم} سے 
\begin{align*}
\tau&=\phi \Re\\
N i &= \phi \left(\frac{l}{\mu_0 A} \right)\\
\frac{\phi}{A}&=B=\frac{ N i \mu_0}{l}
\end{align*}
لکھ کر درج ذیل حاصل ہو گا۔
\begin{align*}
0.1&=\frac{100 \times i \times 4 \pi  10^{-7}}{0.001}\\
i&=\frac{0.1 \times 0.001}{100 \times 4 \pi  10^{-7}}=\SI{0.79577}{\ampere}
\end{align*}
\عددیء{i=\SI{0.79577}{\ampere}} برقی رو  خلائی درز میں \عددیء{B=\SI{0.1}{\tesla}} کثافتِ مقناطیسی بہاو پیدا کرے گا۔
\انتہا{مثال}
%
\حصہ{مقناطیسی دور حصہ دوم}
شکل \حوالہ{شکل_مقناطیسی__سادہ_مقناطیسی_دور_بغیر_درز} میں ایک سادہ مقناطیسی نظام دکھایا گیا ہے جس میں قالب کے مقناطیسی مستقل کو محدود تصور کرتے ہیں۔مقناطیسی دباو  \عددیء{\tau=N i} مقناطیسی قالب میں مقناطیسی بہاو \عددیء{\phi_c} پیدا کرتا ہے۔ قالب کا رقبہ عمودی تراش \عددیء{A_c}  ہر مقام پر  یکساں ہے اور قالب  کی اوسط لمبائی \عددیء{l_c} ہے۔ قالب میں مقناطیسی بہاو  کا رخ
  \اصطلاح{فلیمنگ کے دائیں ہاتھ قانون}\حاشیہد{فلیمنگ!دایاں ہاتھ قانون}  سے\حاشیہب{Fleming's right hand rule} معلوم کیا جا سکتا ہے۔اس قانون کو دو طریقوں سے بیان کیا جا سکتا ہے۔
\begin{itemize}
\item
اگر ایک لچھے کو دائیں ہاتھ سے یوں پکڑا  جائے کہ ہاتھ کی چار انگلیاں لچھے میں برقی رو کے رخ لپٹی  ہوں تب انگوٹھا اُس مقناطیسی بہاو کے رخ ہو گا جو اس برقی رو کی وجہ سے وجود میں آیا ہو۔
\item
اگر ایک تار جس میں برقی رو کا گزر ہو کو دائیں ہاتھ سے یوں پکڑا جائے کہ انگوٹھا  برقی رو  کے رخ ہو تب باقی چار انگلیاں اُس مقناطیسی  بہاو کے رخ لپٹی ہوں گی  جو اس برقی رو کی وجہ سے  پیدا ہو گا۔
\end{itemize}

ان دو بیانات میں پہلا بیان  لچھے میں مقناطیسی بہاو کا رخ معلوم کرنے کے لئے زیادہ آسان ثابت ہوتا ہے جبکہ  سیدھی تار کے گرد مقناطیسی بہاو کا رخ دوسرے بیان سے زیادہ آسانی سے معلوم کیا جا سکتا ہے۔
\begin{figure}
\centering
\begin{tikzpicture}
\def\height{2};
\def\width{1.5};
\def\thick{0.4};
\def\depthX{0.2};
\def\depthY{0.2};
\def\gap{0.05};
%grid
%\draw[gray,thick](0,0) grid (5,3);
%\draw[gray,thin,xstep=0.1,ystep=0.1](0,0) grid (5,3);
%going clockwise from origin
\draw(0,0)--++(0,\height)--++(\width,0)--++(0,-\height)--cycle;
\draw(0,0)++(\thick,\thick)--++(0,\height-2*\thick)--++(\width-2*\thick,0)--++(0,-\height+2*\thick)--cycle;
%
\draw(\thick,\thick)--++(\depthX,\depthY) --++(0,\height-2*\thick-\depthY);
\draw(\thick,\thick)--++(\depthX,\depthY) --++(\width-2*\thick-\depthX,0);
\draw(0,\height)--++(\depthX,\depthY)--++(\width,0)--++(-\depthX,-\depthY);
\draw(\width,0)--++(\depthX,\depthY)--++(0,\height)--++(-\depthX,-\depthY);
%flux
\draw[gray,-latex](1.1,1.8)node[left,black]{$\phi_c$}--++(0.2,0)--++(0,-\height+\thick)--++(-\width+\thick,0)--++(0,\height-\thick)--++(0.3,0);
%winding
\draw (0.6,1.4) to [out=45,in=0] (0.2,1.5) to [short,i_<=$i$] (-1,1.5) node[left]{$+$};
\foreach \l in {1.4,1.2,1}{
\draw (0,\l) to [out=-135,in=45] (0.6,\l-0.2);
}
\draw (0,0.8) to (-1,0.8)node[left]{$-$};
%turns
\node at (0,1.15)[left]{$\tau=N i$};
%urdu coil
\draw[thin,<-](0.3,0.9) to [out=-90,in=60] (-0.7,-0.3)node[below]{\RL{چکر کا لچھا}$\,N$};
%dimensions
\draw[stealth-stealth] (1.1,-0.1)--++(0.4,0)node[below,pos=0.5]{$b$};
\draw[stealth-stealth](1.5+0.1,-0.1)--++(0.2,0.2)node [pos=0.4,right]{$w$};
%cross sectional area
\draw[gray](1.1,1)--++(\thick,0)--++(\depthX,\depthY)--++(-\thick,0)--cycle;
\draw[gray,<-] (1.6,1.2) to [out=90,in=-90](2.4,1.7)node[above right,black]{$A_c=b w$};
\draw[gray,<-](1.3,0.6) to [out=0,in=-180] (2.7,1)node [right,black]{\RL{اس لکیر پر اوسط لمبائی $\,l_c\,$ ہے۔}};
\end{tikzpicture}%
\caption{سادہ مقناطیسی دور۔}
\label{شکل_مقناطیسی__سادہ_مقناطیسی_دور_بغیر_درز}
\end{figure}

قالب میں مقناطیسی بہاو  گھڑی وار ہے۔ مقناطیسی بہاو \عددی{\phi} کو  شکل \حوالہ{شکل_مقناطیسی__سادہ_مقناطیسی_دور_بغیر_درز} میں  ہلکی سیاہی کے تیر دار لکیر  سے ظاہر کیا گیا ہے۔ قالب کی ہچکچاہٹ 
\begin{align*}
\Re_c&=\frac{l_c}{\mu_c A_c}
\end{align*}
لکھتے ہوئے مقناطیسی بہاو 
\begin{align*}
\phi_c&=\frac{\tau}{\Re_c}=N i \left(\frac{\mu_c A_c}{l_c} \right)
\end{align*}
ہو گا۔یوں  تمام نا معلوم متغیرات حاصل ہو چکے۔
%
\ابتدا{مثال}
شکل \حوالہ{شکل_مقناطیسی__درز_اور_ہچکچاہٹ}  میں ایک مقناطیسی قالب دکھایا گیا ہے جس کی معلومات درج ذیل ہیں۔
\begin{align}
\text{قالب}= \left\{ 
  \begin{array}{l l}
  h=\SI{20}{\centi\meter} & m=\SI{10}{\centi \meter}\\
 n=\SI{8}{\centi\meter} & w=\SI{2}{\centi \meter}\\
 l_a=\SI{1}{\milli\meter} & \mu_r =\num{40000} \\
 \end{array} \right.
\end{align}
قالب اور خلائی درز کی ہچکچاہٹیں تلاش کریں۔
\begin{figure}
\centering
\begin{tikzpicture}
\def\height{2};
\def\width{1.5};
\def\thick{0.4};
\def\depthX{0.2};
\def\depthY{0.2};
\def\gap{0.05};
%grid
%\draw[gray,thick](0,0) grid (12,3);
%\draw[gray,thin,xstep=0.1,ystep=0.1](0,0) grid (12,3);
%going clockwise from origin
\draw(0,0)--++(0,\height)--++(\width,0)--++(0,-0.5*\height+\gap)--++(-\thick,0)--++(0,0.5*\height-\gap-\thick)--++(-\width+2*\thick,0)--++(0,-\height+2*\thick)--++(\width-2*\thick,0)--++(0,0.5*\height-\thick-\gap)--++(\thick,0)--++(0,-0.5*\height+\gap)--cycle;
%
\draw(\thick,\thick)--++(\depthX,\depthY) --++(0,\height-2*\thick-\depthY);
\draw(0.4,0.4)--++(\width-2*\thick-\depthX,0);
\draw(0,\height)--++(\depthX,\depthY)--++(\width,0)--++(-\depthX,-\depthY);
\draw(\width,\height)++(\depthX,\depthY)--++(0,-0.5*\height+\gap)--++(-\depthX,-\depthY);
\draw(\width,0)--++(\depthX,\depthY)--++(0,0.5*\height-\gap)--++(-\depthX,-\depthY);
\draw(\thick+\depthX,\thick+\depthY)--++(\width-2*\thick-\depthX,0);
%gap
\draw(1.7,1.155)--++(-0.1,0);
\draw(1.1,0.95)--++(0.1,0.1);

%gap
\draw[gray](1.8,1.15)--++(0.3,0);
\draw[gray](1.8,1.25)--++(0.3,0);
\draw[->](2.2,1.7)node[right]{$l_a$}--++(-0.2,-0.2)--++(0,-0.25);
\draw[->](2,1)--++(0,0.15);
%flux
\draw[gray](1.1,1.8)--++(0.2,0)--++(0,-\height+\thick)--++(-\width+\thick,0)--++(0,\height-\thick)--++(0.5,0)--cycle;
%dimensions
\draw[stealth-stealth] (1.1,-0.1)--++(0.4,0)node[below,pos=0.5]{$b$};
\draw[stealth-stealth](1.5+0.1,-0.1)--++(0.2,0.2)node [pos=0.4,right]{$w$};
\draw[stealth-stealth] (-0.1,0)--++(0,\height) node [left,pos=0.5]{$h$};
\draw[stealth-stealth] (-0.3,0)--++(0,0.4) node [left,pos=0.5]{$b$};
\draw (-0.05,0)--++(-0.4,0);
\draw (-0.2,0.4)--++(-0.2,0);
\draw[stealth-stealth](\thick,-0.1)--++(\width-2*\thick,0)node[pos=0.5,below]{$n$};
\draw[stealth-stealth](0,-0.7)--++(\width,0)node[pos=0.5,below]{$m$};
\draw[gray,stealth-](0.8,1.8) to [out=90,in=180] (2,2.7) node[right,black]{$\,l_c\,$ \RL{مرکز کی لمبائی}};
%equations
\draw (4,0) node[above right]{$
\begin{aligned}
A_a&=A_c=bw\\
b&=\frac{m-n}{2}\\
l_c&=2(h-b)+2(m-b)-l_a
\end{aligned}
$};
\end{tikzpicture}%
\caption{خلائی درز اور قالب کے ہچکچاہٹ۔}
\label{شکل_مقناطیسی__درز_اور_ہچکچاہٹ}
\end{figure}

حل:\quad
\begin{align*}
b&=\frac{m-n}{2}=\frac{0.1-0.08}{2}=\SI{0.01}{\meter}\\
A_a&=A_c=bw=0.01 \times 0.02=\SI{0.0002}{\square \meter}\\
l_c&=2(h-b)+2(m-b)-l_a\\
&=2(0.2-0.01)+2(0.1-0.01)-0.001=\SI{0.559}{\meter}
\end{align*}
%
\begin{align*}
\Re_c&=\frac{l_c}{\mu_r \mu_0 A_c}=\frac{0.559}{40000 \times 4 \pi 10^{-7} \times 0.0002}=\SI{55605}{\ampere \cdot t \per \weber}\\
\Re_a&=\frac{l_a}{\mu_0 A_a}=\frac{0.001}{4 \pi 10^{-7} \times 0.0002}=\SI{3978874}{\ampere \cdot t \per \weber}
\end{align*}
قالب کی لمبائی خلائی درز کی لمبائی سے \عددیء{359} گنا زیادہ ہونے کے باوجود خلائی درز کی ہچکچاہٹ قالب کی ہچکچاہٹ سے \عددیء{72} گنا زیادہ ہے۔یوں  \عددیء{\Re_a  \gg \Re_c} ہو گا۔
\انتہا{مثال}
%
\ابتدا{مثال}
شکل  \حوالہ{شکل_مقناطیسی_دور_سادہ_گھومتا_مشین} سے رجوع کریں۔خلائی درز \عددیء{5} ملی میٹر لمبا ہے اور گھومتے حصہ پر \عددیء{1000} چکر ہیں۔خلائی درز میں \عددیء{\SI{0.95}{\tesla}} کثافتِ مقناطیسی بہاو حاصل کرنے کی خاطر درکار برقی رو معلوم کریں۔
\begin{figure}
\centering
%\includegraphics{figMagneticCircuitsSimpleRotatingMachineOutline}
\begin{tikzpicture}
\pgfmathsetmacro{\radI}{1} 
\pgfmathsetmacro{\radO}{1.3} 
\pgfmathsetmacro{\radAv}{(\radI+\radO)/2}
\pgfmathsetmacro{\rad}{0.8} 
\pgfmathsetmacro{\delTheta}{20} 
\pgfmathsetmacro{\yUP}{\rad*cos(\delTheta) } 
%machine stator dimensions
\draw (0,0) circle (\radI);
\draw (0,0) circle (\radO);

%==================================================
%the scope has been added to rotate everything by 35 degrees
%\begin{scope}[rotate=35]
%rotor at zero degrees
\draw (0,0)++(-\delTheta:\rad) arc (-\delTheta:\delTheta:\rad)--(180-\delTheta:\rad) arc (180-\delTheta:180+\delTheta:\rad)--cycle;
%winding on rotor
\foreach \x in {0.2,-0.2}
{
\draw (\x,0.28) to [out=135,in=-45]  (\x-0.2,-0.28);
}
%winding end connections
\draw(0.4,-0.28) to [out=-45,in=-90] (0.5,0) to (0.5,0.3) to [short,i_<=$i$](0.5,0.6);
\draw(-0.6,0.28) --(-0.6,0.6);
\draw(0,-0.5)node{$N$};
%flux upper half
\draw[gray](0.9*\delTheta:\radAv) to [out=-70,in=0] (0.4*\delTheta:0.9*\rad);
\draw[gray,-<-=0.6] (0.4*\delTheta:0.9*\rad)--(180-0.4*\delTheta:0.9*\rad);
\draw[gray](180-0.4*\delTheta:0.9*\rad) to [out=180,in=-110] (180-0.9*\delTheta:\radAv);
\draw[gray,->-=0.5](0.9*\delTheta:\radAv) arc (0.9*\delTheta:180-0.9*\delTheta:\radAv);
%flux lower half
\begin{scope}[rotate=180]
%
\draw[gray](0.9*\delTheta:\radAv) to [out=-70,in=0] (0.4*\delTheta:0.9*\rad);
\draw[gray,->-=0.43] (0.4*\delTheta:0.9*\rad)--(180-0.4*\delTheta:0.9*\rad);
\draw[gray](180-0.4*\delTheta:0.9*\rad) to [out=180,in=-110] (180-0.9*\delTheta:\radAv);
\draw[gray,-<-=0.5](0.9*\delTheta:\radAv) arc (0.9*\delTheta:180-0.9*\delTheta:\radAv);
\end{scope}
%urdu
\draw[gray,<-](-35:\radO) to [out=-35,in=180] (2,-1)node[black][right]{\RL{ساکن حصہ }};
\draw[gray,<-](0,0.3) to [out=90,in=0] (-2,0.5) node[left,black]{\RL{گھومتا حصہ}};
%
\draw[gray,decorate,decoration={brace,mirror}]  (1.5,-0.3) -- node[right,black] {\RL{خلائی درز میں مقناطیسی بہاو}} (1.5,0.3);
\draw[gray,<-] (30:\radAv) to [out=30,in=180] (1.5,1)node[right,black]{\RL{مقناطیسی بہاو}};
\draw[->] (180:0.7*\rad)--(180:\rad);
\draw[<-](180:\radI)--(180:1.3*\radO)--++(-0.3,-0.3)node[below]{$l_a$};
%\end{scope}
%===========================================
\end{tikzpicture}%
\caption{سادہ گھومنے والا مشین}
\label{شکل_مقناطیسی_دور_سادہ_گھومتا_مشین}
\end{figure}

حل:\quad
اس شکل میں گھومتے مشین، مثلاً موٹر، کی ایک سادہ صورت دکھائی گئی ہے۔ ایسی مشینوں کا بیرونی حصہ ساکن رہتا ہے لہٰذا اس حصے  کو مشین کا \اصطلاح{ساکن حصہ}\فرہنگ{ساکن حصہ}\حاشیہب{stator}\فرہنگ{stator} کہتے ہیں۔ساکن حصے کے اندر مشین کا  گھومتا حصہ پایا جاتا ہے لہٰذا اس حصے کو مشین کا \اصطلاح{گھومتا حصہ}\فرہنگ{گھومتا حصہ}\حاشیہب{rotor}\فرہنگ{rotor} کہتے ہیں۔ اس مثال میں ان دونوں  حصوں (قالب) کا  \عددیء{\mu_r=\infty}  تصور کیا گیا ہے لہٰذا ان کی ہچکچاہٹ صفر ہو گی۔ مقناطیسی بہاو  کو ہلکی سیاہی کی لکیر سے ظاہر کیا گیا ہے۔ مقناطیسی بہاو کی ایک مکمل چکر کے دوران مقناطیسی بہاو دو خلائی درزوں  سے گزرتا ہے۔ یہ دو خلائی درز ہر لحاظ سے ایک دوسرے  جیسے ہیں لہٰذا ان دونوں خلائی درز کی ہچکچاہٹ بھی ایک دوسرے کے برابر ہو گی۔مزید دونوں خلائی درزوں کی ہچکچاہٹ سلسلہ وار ہیں۔شکل \حوالہ{شکل_مقناطیسی_دور_سادہ_گھومتا_مشین} میں مقناطیسی بہاو کو گھومتے حصہ، ساکن حصہ اور دو خلائی درزوں  سے گزرتا ہوا دکھایا گیا ہے۔خلائی درز کی لمبائی \عددیء{l_a}، قالب کے رقبہ \عددی{A_c} کی اضلاع سے بہت کم ہے لہٰذا خلائی درز کا عمودی رقبہ تراش \عددیء{A_a}  گھومتے حصہ کے رقبہ تراش  کے برابر تصور کیا جائے گا۔

یوں \عددی{A_a=A_c} لیتے ہوئے ایک خلائی درز کی ہچکچاہٹ
\begin{align*}
\Re_a=\frac{l_a}{\mu_0 A_a}=\frac{l_a}{\mu_0 A_c}
\end{align*}
اور دو سلسلہ وار خلائی درزوں کی کل ہچکچاہٹ درج ذیل ہو گی۔
\begin{align*}
\Re_s=\Re_a+\Re_a=\frac{2 l_a}{\mu_0 A_c}
\end{align*}
خلائی درز میں مقناطیسی بہاو \عددیء{\phi_a} اور کثافتِ مقناطیسی بہاو \عددیء{B_a} درج ذیل ہوں گے۔
\begin{align*}
\phi_a&=\frac{\tau}{\Re_s}=\left(N i \right) \left (\frac{\mu_0 A_c}{2 l_a} \right)\\
B_a&=\frac{\phi_a}{A_a}=\frac{\mu_0 N i}{2 l_a}
\end{align*}
دی گئی معلومات پر کرتے ہوئے درج ذیل حاصل ہو گا۔
\begin{align*}
0.95&=\frac{4 \pi 10^{-7} \times 1000 \times i}{2 \times 0.005}\\
i&=\frac{0.95 \times 2 \times 0.005}{ 4 \pi 10^{-7} \times 1000}=\SI{7.56}{\ampere}
\end{align*}
روایتی موٹروں اور جنریٹروں کی خلاء میں تقریباً ایک ٹسلا کثافتِ مقناطیسی بہاو ہوتا ہے۔
\انتہا{مثال}

\حصہ{خود امالہ  ، مشترکہ امالہ  اور توانائی}
وقت کے ساتھ بدلتا مقناطیسی میدان برقی دباو پیدا کرتا ہے جس کو \اصطلاح{قانون فیراڈے}\فرہنگ{فیراڈے!قانون}\حاشیہب{Faraday's law}\فرہنگ{Faraday's law}
\begin{align*}
\oint_C\kvec{E}\cdot\dif \kvec{l}=-\frac{\dif}{\dif t}\int_S\kvec{B}\cdot\dif \kvec{S}
\end{align*}
 سے حاصل کیا جا سکتا ہے\حاشیہد{مائکل فیراڈے انگلستانی سائنسدان تھے جنہوں نے محرک برقی دباو دریافت کی۔}۔یہ مساوات کہتی ہے کہ کسی بند راہ کی ہمراہ مقناطیسی سمتی میدان \عددی{\kvec{E}} کا ارتفاعی تکمل اس راہ کے ارتباط بہاو کے  (وقت کے ساتھ) تفرق کے برابر ہو گا۔ برقی ادوار، مثلاً شکل \حوالہ{شکل_مقناطیسی_بہاو_تبدیلی_اور_دباو}-ا،  میں مستعمل برقی تاروں  کی ہمراہ \عددی{\kvec{E}} قابل نظر انداز ہوتا ہے لہٰذا اس مساوات کا بایاں ہاتھ تاروں کے سروں پر  \اصطلاح{امالی برقی دباو}\فرہنگ{امالی برقی دباو}\حاشیہب{induced voltage}\فرہنگ{induced voltage} \عددی{e} کے منفی کے برابر ہو گا۔ساتھ ہی مساوات کے دائیں ہاتھ تکمل میں بہاو کا بیشتر حصہ قالب کے اندر بہاو \عددی{\phi} پر مشتمل ہو گا۔ چونکہ لچھا (اور بند راہ)  اس قالب کے گرد \عددی{N} چکر کاٹتا ہے لہٰذا  یہ مساوات درج ذیل صورت اختیار کرتی ہے۔ 
\begin{align}\label{مساوات_مقناطیسی_دور_فیراڈے_قانون}
e=N \frac{\partial \phi}{\partial t} =\frac{\partial \lambda}{\partial t}
\end{align}

اس طرح  شکل \حوالہ{شکل_مقناطیسی_بہاو_تبدیلی_اور_دباو}-ا  کے قالب میں مقناطیسی بہاو \عددی{\phi} کی تبدیل کی بنا لچھے میں برقی دباو \عددی{e} پیدا ہو گا جو لچھے کے سروں پر نمودار ہو گا۔
%
\begin{figure}
\centering
\begin{subfigure}{0.45\textwidth}
\centering
\begin{circuitikz}
\def\height{2};
\def\width{1.5};
\def\thick{0.4};
\def\depthX{0.2};
\def\depthY{0.2};
\def\gap{0.05};
%flux
\draw[gray,latex-](\width-\thick/2,1.8-0.25) to [out=120,in=0](\width/2,1.8)node[fill=white,text=black]{$\phi$} to [out=180,in=70](\thick/2,1.8-0.25);
%going clockwise from origin
\draw(0,0)--++(0,\height)--++(\width,0)--++(0,-\height)--cycle;
\draw(0,0)++(\thick,\thick)--++(0,\height-2*\thick)--++(\width-2*\thick,0)--++(0,-\height+2*\thick)--cycle;
%
\draw(\thick,\thick)--++(\depthX,\depthY) --++(0,\height-2*\thick-\depthY);
\draw(\thick,\thick)--++(\depthX,\depthY) --++(\width-2*\thick-\depthX,0);
\draw(0,\height)--++(\depthX,\depthY)--++(\width,0)--++(-\depthX,-\depthY);
\draw(\width,0)--++(\depthX,\depthY)--++(0,\height)--++(-\depthX,-\depthY);
%\draw[gray,-latex](\width/2,1.8)node[black]{$\phi$}++(-0.3,0) to [out=180,in=70]++(-0.25,-0.25);
%winding
\draw (0.6,1.4) to [out=45,in=0] (0.2,1.5) to [short] (-1,1.5) node[left]{$+$};
\foreach \l in {1.4,1.2,1}{
\draw (0,\l) to [out=-135,in=45] (0.6,\l-0.2);
}
\draw (0,0.8) to (-1,0.8)node[left]{$-$};
\node at (-1,1.15)[left]{$e$};
\draw($(0,1.5)!0.5!(0,0.8)$)node[left]{$N$};
\end{circuitikz}%
\caption{}
\end{subfigure}\hfill
\begin{subfigure}{0.45\textwidth}
\centering
\begin{circuitikz}
\def\height{2};
\def\width{1.5};
\def\thick{0.4};
\def\depthX{0.2};
\def\depthY{0.2};
\def\gap{0.05};
%flux
\draw[gray,-latex](\width-\thick/2,1.8-0.25) to [out=120,in=0](\width/2,1.8)node[fill=white,text=black]{$\phi'$} to [out=180,in=70](\thick/2,1.8-0.25);
%going clockwise from origin
\draw(0,0)--++(0,\height)--++(\width,0)--++(0,-\height)--cycle;
\draw(0,0)++(\thick,\thick)--++(0,\height-2*\thick)--++(\width-2*\thick,0)--++(0,-\height+2*\thick)--cycle;
%
\draw(\thick,\thick)--++(\depthX,\depthY) --++(0,\height-2*\thick-\depthY);
\draw(\thick,\thick)--++(\depthX,\depthY) --++(\width-2*\thick-\depthX,0);
\draw(0,\height)--++(\depthX,\depthY)--++(\width,0)--++(-\depthX,-\depthY);
\draw(\width,0)--++(\depthX,\depthY)--++(0,\height)--++(-\depthX,-\depthY);
%\draw[gray,-latex](\width/2,1.8)node[black]{$\phi$}++(-0.3,0) to [out=180,in=70]++(-0.25,-0.25);
%winding
\draw (0.6,1.4) to [out=45,in=0] (0.2,1.5) to [short] (-1,1.5)coordinate(kT) node[left]{$+$};
\foreach \l in {1.4,1.2,1}{
\draw (0,\l) to [out=-135,in=45] (0.6,\l-0.2);
}
\draw (0,0.8) to (-1,0.8)coordinate(kB)node[left]{$-$};
\draw(kT)to [short]++(0,0.5)--++(-1,0) to [resistor,i>_={$i$}]++(0,-2)-|(kB);
\node at (-1,1.15)[left]{$e$};
\draw($(0,1.5)!0.5!(0,0.8)$)node[left]{$N$};
\end{circuitikz}%
\caption{}
\end{subfigure}%
\caption{قالب میں مقناطیسی بہاو کی تبدیلی لچھے میں برقی دباو پیدا کرتی ہے۔}
\label{شکل_مقناطیسی_بہاو_تبدیلی_اور_دباو}
\end{figure}

امالی برقی دباو کو منبع برقی دباو تصور کریں۔

امالی برقی دباو  کا رخ تعین کرنے کی خاطر  لچھے کے سروں کو \اصطلاح{قصر دور}\فرہنگ{ْسر دور}\حاشیہب{short circuit}   کریں۔لچھے میں پیدا برقی رو اُس رخ  ہو گا جو مقناطیسی بہاو کی تبدیلی کو روکے۔

 فرض کریں شکل \حوالہ{شکل_مقناطیسی_بہاو_تبدیلی_اور_دباو}-ا میں بہاو \عددی{\phi}  گھڑی وار ہے  اور  بہاو کی مقدار بڑھ رہی ہے۔بہاو میں تبدیلی کو روکنے کی خاطر  بہاو \عددی{\phi'} پیدا کرنا ہو گا جو لچھے کا بالائی سر مثبت  ہونے سے  ہو گا۔شکل \حوالہ{شکل_مقناطیسی_بہاو_تبدیلی_اور_دباو}-ب میں لچھے کے سروں کے بیچ مزاحمت نسب کیا گیا ہے۔ لچھے کو منبع دباو تصور کرتے ہوئے  آپ دیکھ سکتے ہیں کہ مزاحمت میں رو کا رخ قالب میں گھڑی کے مخالف رخ بہاو \عددی{\phi'} پیدا کرے گا۔

قالب میں مقناطیسی بہاو \عددیء{\phi}، قالب پر لپیٹے گئے لچھے کے تمام چکروں، \عددیء{N}، کے اندر سے گزرتا ہے۔\عددیء{N \phi} کو لچھے کا \اصطلاح{ارتباط بہاو}\فرہنگ{ارتباط بہاو}\حاشیہب{flux linkage} \عددیء{\lambda}  کہتے ہیں جس کی اکائی \اصطلاح{ویبر-چکر}\فرہنگ{ویبر-چکر}\حاشیہب{weber-turn}  ہے۔
\begin{align}
\lambda=N\phi
\end{align}
جن مقناطیسی ادوار میں مقناطیسی مستقل \عددیء{\mu}  کو اٹل مقدار تصور کیا جا سکے یا جن میں خلائی درز کی ہچکچاہٹ قالب کی ہچکچاہٹ سے بہت زیادہ ہو، \عددیء{\Re_a \gg \Re_c}،  ان میں لچھے کی  \اصطلاح{امالہ}\فرہنگ{امالہ}\حاشیہب{inductance}\فرہنگ{inductance} \عددیء{L}  کی تعریف درج ذیل مساوات دیتی ہے۔
\begin{align}\label{مساوات_مقناطیسی_دور_خود_امالہ_تعریف}
L=\frac{\lambda}{i}
\end{align}

امالہ کی اکائی ویبر-چکر فی ایمپیئر ہے جس کو \اصطلاح{ہینری}\فرہنگ{Henry}\حاشیہب{Henry}\فرہنگ{Henry} \عددیء{H} کا نام\حاشیہد{امریکی سائنسدان جوزف ہینری جنہوں نے مائکل فیراڈے سے علیحدہ  طور پر محرک برقی دباو دریافت کی} دیا گیا ہے۔ مساوات \حوالہ{مساوات_مقناطیسی_دور_خود_امالہ_تعریف} میں \عددی{\lambda=N\phi}،  \عددی{\phi=B_cA_c}  اور \عددی{\phi=\tfrac{Ni}{\Re}} پر کرتے ہوئے درج ذیل حاصل ہو گا
\begin{align}
L=\frac{N \phi}{i}=\frac{N B_c A_c}{i}=\frac{N^2 \mu_0 A_a}{l_a}
\end{align}
جہاں قالب کا رقبہ عمودی تراش \عددی{A_c} اور درز کا رقبہ عمودی تراش \عددی{A_a} ایک دوسرے کے برابر لیے گئے ہیں۔
%
\ابتدا{مثال}\شناخت{مثال_مقناطیسی_امالہ_الف}
شکل \حوالہ{شکل_مثال_مقناطیسی_امالہ_الف} میں \عددیء{b=\SI{5}{\centi \meter},w=\SI{4}{\centi\meter},l_a=\SI{3}{\milli \meter}} جبکہ لچھے کے \عددیء{1000} چکر اور قالب کی اوسط لمبائی \عددیء{l_c=\SI{30}{\centi\meter}} ہے۔درج ذیل دو صورتوں میں لچھے کی امالہ تلاش کریں۔
\begin{itemize}
\item
قالب کا \عددیء{\mu_r = \infty} ہے۔
\item
قالب کا \عددیء{\mu_r = 500} ہے۔
\end{itemize}
\begin{figure}
\centering
\begin{tikzpicture}
\def\height{2};
\def\width{1.5};
\def\thick{0.4};
\def\depthX{0.2};
\def\depthY{0.2};
\def\gap{0.05};
\pgfmathsetmacro{\widthX}{0.25}  
\pgfmathsetmacro{\widthY}{0.25}  
\coordinate(O) at (-0.5,0);
%grid
%\draw[gray,thick](0,0) grid (12,3);
%\draw[gray,thin,xstep=0.1,ystep=0.1](0,0) grid (12,3);
%going clockwise from origin
\draw(0,0)--++(0,\height)--++(\width,0)--++(0,-0.5*\height+\gap)--++(-\thick,0)--++(0,0.5*\height-\gap-\thick)--++(-\width+2*\thick,0)--++(0,-\height+2*\thick)--++(\width-2*\thick,0)--++(0,0.5*\height-\thick-\gap)--++(\thick,0)--++(0,-0.5*\height+\gap)--cycle;
%
\draw(\thick,\thick)--++(\depthX,\depthY) --++(0,\height-2*\thick-\depthY);
\draw(0.4,0.4)--++(\width-2*\thick-\depthX,0);
\draw(0,\height)--++(\depthX,\depthY)--++(\width,0)--++(-\depthX,-\depthY);
\draw(\width,\height)++(\depthX,\depthY)--++(0,-0.5*\height+\gap)--++(-\depthX,-\depthY);
\draw(\width,0)--++(\depthX,\depthY)--++(0,0.5*\height-\gap)--++(-\depthX,-\depthY);
\draw(\thick+\depthX,\thick+\depthY)--++(\width-2*\thick-\depthX,0);
%gap
\draw(1.7,1.155)--++(-0.1,0);
\draw(1.1,0.95)--++(0.1,0.1);
%winding
\draw (0.6,1.4) to [out=45,in=0] (0.2,1.5) to [short,i_<=$i$] (-1,1.5) node[left]{$+$};
\foreach \l in {1.4,1.2,1}{
\draw (0,\l) to [out=-135,in=45] (0.6,\l-0.2);
}
\draw (0,0.8) to (-1,0.8)node[left]{$-$};
%turns
\node at (0,1.15)[left]{$\tau=N i$};
%gap
\draw[gray](1.8,1.15)--++(0.3,0);
\draw[gray](1.8,1.25)--++(0.3,0);
\draw[-stealth](2.2,1.7)node[right]{$l_a$}--++(-0.2,-0.2)--++(0,-0.25);
\draw[-stealth](2,1)--++(0,0.15);
%flux
\draw[-latex,gray](1.1,1.8)node[left,black]{$\phi$}--++(0.2,0)--++(0,-\height+\thick)--++(-\width+\thick,0)--++(0,\height-\thick)--++(0.5,0);
%dimensions
\draw[stealth-stealth] (1.1,-0.1)--++(0.4,0)node[below,pos=0.5]{$b$};
\draw[stealth-stealth](1.5+0.1,-0.1)--++(0.2,0.2)node [pos=0.4,right]{$w$};
\end{tikzpicture}%
\caption{امالہ (مثال \حوالہ{مثال_مقناطیسی_امالہ_الف})}
\label{شکل_مثال_مقناطیسی_امالہ_الف}
\end{figure}

حل:\quad
(ا) \quad 
 قالب کے \عددیء{\mu_r=\infty}  کی بنا قالب کی ہچکچاہٹ قابل نظرانداز ہو گی لہٰذا  امالہ درج ذیل ہو گا۔
\begin{align*}
L&=\frac{N^2 \mu_0 w b}{l_a}\\
&=\frac{1000^2 \times 4 \pi 10^{-7} \times 0.04 \times 0.05}{0.003}\\
&=\SI{0.838}{\henry}
\end{align*}
(ب) \quad
 \عددیء{\mu_r=500} کی صورت میں قالب کی ہچکچاہٹ قابل نظر انداز نہیں ہو گی۔خلاء اور قالب کی ہچکچاہٹ  دریافت کرتے ہیں۔
\begin{align*}
\Re_a&=\frac{l_a}{\mu_0 w b}=\frac{0.003}{4\pi 10^{-7} \times 0.04 \times 0.05}=\SI{1193662}{\ampere \cdot t \per \weber}\\
\Re_c&=\frac{l_c}{\mu_r \mu_0 w b}=\frac{0.3}{500 \times 4\pi 10^{-7} \times 0.04 \times 0.05}=\SI{238701}{\ampere \cdot t \per \weber}
\end{align*}
یوں بہاو، ارتباط اور امالہ درج ذیل ہوں گے۔
\begin{align*}
\phi&=\frac{N i}{\Re_a+\Re_c}\\
\lambda &= N \phi = \frac{N^2 i}{\Re_a+\Re_c}\\
L&=\frac{\lambda}{i}=\frac{N^2}{\Re_a+\Re_c}=\frac{1000^2}{\num{1193662}+\num{238701}}=\SI{0.698}{\henry}
\end{align*}
\انتہا{مثال}
%
\ابتدا{مثال}
شکل \حوالہ{شکل_مقناطیسی_ادوار_پیچدار_لچھا} میں ایک پیچدار لچھا\فرہنگ{لچھا!پیچدار}\حاشیہب{spiral coil} دکھایا گیا ہے جس کی جسامت درج ذیل ہے۔
\begin{align*}
N=11, r=\SI{0.49}{\meter},l=\SI{0.94}{\meter}
\end{align*}
پیچدار لچھے  کے اندر مقناطیسی بہاو \عددی{\phi} کا بیشتر حصہ محوری رخ ہوتا ہے۔ لچھے کے باہر یہی بہاو پوری کائنات سے گزرتے ہوئے واپس لچھے میں داخل ہوتا ہے۔چونکہ پوری کائنات کا رقبہ عمودی تراش \عددی{A} لامتناہی ہے لہٰذا لچھے کے باہر کثافت مقناطیسی بہاو \عددی{B=\tfrac{\phi}{A}} کی مقدار قابل نظرانداز ہو گی۔لچھے کے اندر محوری رخ مقناطیسی شدت درج ذیل ہو گی۔
\begin{align*}
H=\frac{N i}{l}
\end{align*}
اس لچھے کی خود امالہ حاصل کریں۔
\begin{figure}[h!]
\centering
\begin{tikzpicture}
\pgfmathsetmacro{\radX}{1.8}
 \pgfmathsetmacro{\radY}{0.4}
\pgfmathsetmacro{\thetaStart}{0} 
\pgfmathsetmacro{\thetaEnd}{180}
%grid
%\draw[gray,thick] (-4*\radX,-4*\radX) grid (4*\radX,4*\radX);
%\draw[gray,thin,xstep=0.1,ystep=0.1] (-4*\radX,-4*\radX) grid (4*\radX,4*\radX);
%coil backside
\foreach \y in {0,0.2,0.4,0.6,0.8,1,1.2,1.4,1.6,1.8,2}{
\draw[thick,domain={0:180},variable=\t,smooth,samples=100] plot({\radX*cos(\t)},{\y+0.1/180*\t+\radY*sin(\t)});
}
%coil front
\foreach \y in {0,0.2,0.4,0.6,0.8,1,1.2,1.4,1.6,1.8,2}{
\draw[thick,domain={180:360},variable=\t,smooth,samples=100] plot({\radX*cos(\t)},{\y+0.1/180*\t+\radY*sin(\t)});
}
%end connections
\draw[thick](\radX,0)--(1.4*\radX,0);
\draw[thick](\radX,2+0.1/180*360)--(1.4*\radX,2+0.1/180*360);
%text
\draw[gray,stealth-stealth](1.4*\radX,0+0.05)--(1.4*\radX,2+0.1/180*360-0.05)node[pos=0.5,right]{$l$};
\draw[gray,dashed](0,2+0.1/180*360-0.4)--(0,2+0.1/180*360+0.4);
\draw[gray,-stealth] (0,2+0.1/180*360+0.4)--++(\radX,0)node [above,pos=0.5]{$r$};
\draw(-2,1)node[left]{\RL{$N$ چکر}};
\end{tikzpicture}
\caption{پیچدار لچھا}
\label{شکل_مقناطیسی_ادوار_پیچدار_لچھا}
\end{figure}

حل:
\begin{align*}
B&=\mu_0 H=\frac{\mu_0 N i}{l}\\
\phi&=B  \pi r^2=\frac{\mu_0 N i \pi r^2}{l}\\ 
\lambda&=N \phi =\frac{\mu_0 N^2 i \pi r^2}{l}\\ 
L&=\frac{\lambda}{i}=\frac{\mu_0 N^2 \pi r^2}{l}
\end{align*} 
\عددی{N}، \عددی{r} اور \عددی{l} کی قیمتیں پر کرتے ہوئے درج ذیل امالہ حاصل ہو گا\حاشیہد{یہ پیچدار لچھا میں نے  \عددی{3000} کلو گرام لوہا پگھلانے والی بھٹی میں استعمال کیا ہے۔}۔
\begin{align*}
L=\frac{4 \pi 10^{-7} \times 11^2 \times \pi  \times 0.49^2}{0.94}=\SI{122}{\micro \henry}
\end{align*}
\انتہا{مثال}
%
شکل \حوالہ{شکل_مقناطیسی_ادوار_دو_لچھے_ایک_درز} میں دو لچھوں کا ایک مقناطیسی دور دکھایا گیا ہے۔ ایک لچھے کے چکر  \عددیء{N_1}  اور اس میں برقی رو \عددیء{i_1} ہے،  دوسرا لچھا \عددیء{N_2} چکر کا ہے اور اس میں برقی  رو \عددیء{i_2} ہے۔ دونوں لچھوں میں مثبت برقی رو قالب میں ایک جیسے  رخ  مقناطیسی دباو  پیدا کرتے ہیں۔ اگر قالب کا \عددی{\Re_c} قابل نظرانداز ہو تب مقناطیسی بہاو \عددیء{\phi}درج ذیل ہو گا۔
\begin{align}
\phi=\left (N_1 i_1 +N_2 i_2 \right ) \frac{\mu_0 A_a}{l_a}
\end{align} 
%
\begin{figure}[h!]
\centering
\begin{tikzpicture}
%grid
%\draw[gray,thick] (0,0) grid (12,2);
%\draw[gray,thin,xstep=0.1,ystep=0.1] (0,0) grid (12,2);
\def\h{2}
\def\w{2.4}
\def\t{0.4}
\def\g{0.2}
%core
\draw (0,0)--++(0,\h)--++(\w/2-\g/2,0)--++(0,-\t)--++(-\w/2+\g/2+\t,0)--++(0,-\h+\t+\t)--++(\w-\t-\t,0)--++(0,\h-\t-\t)--++(-\w/2+\t+\g/2,0)--++(0,\t)--++(\w/2-\g/2,0)--++(0,-\h)--cycle;
%flux
\draw[gray,->-=0.86](\t/2,\t/2)--++(0,\h-\t)--++(\w-\t,0)--++(0,-\h+\t)--cycle;
%left hand coils
\foreach \y in {0.7,0.9,1.1,1.3}{
\draw (0,\y) to [out=-135,in=45](0.4,\y-0.1);
}
%end connection
\draw (0.4,1.4) to [out=45,in=0] (0.2,1.45);
\draw(0.2,1.45) to ++(-0.3,0) to [short,i_<=$i_1$] ++(-0.3,0);
\draw (0,0.5)--++(-0.4,0);
%right hand coils
\foreach \y in {0.7,0.9,1.1,1.3}{
\draw (\w,\y) to [out=-45,in=135](\w-\t,\y-0.1);
}
%end connection
\draw (\w-\t,1.4) to [out=135,in=0] (\w-\t/2,1.45);
\draw(\w-\t/2,1.45) to ++(0.3,0) to [short,i<=$i_2$] ++(0.3,0);
\draw (\w,0.5)--++(0.4,0);
%text
\draw(\t,\h/2) node[right]{$N_1$};
\draw(\w-\t,\h/2) node[left]{$N_2$};
\draw(0,\h/2) node[left]{$\lambda_1$};
\draw(\w,\h/2) node[right]{$\lambda_2$};
%gap
\draw (\w/2-\g/2,\h+0.1)--++(0,0.2);
\draw (\w/2+\g/2,\h+0.1)--++(0,0.2);
\draw[->](\w/2-\g/2-0.2,\h+0.2)--++(0.2,0);
\draw[<-](\w/2+\g/2,\h+0.2)--++(0.2,0)--++(0.1,0.1)node[above right]{$l_a$};
\draw[->](\t+0.2,\h-\t-0.2)--++(0,0.2);
\draw[<-](\t+0.2,\h)--++(0,0.2)--++(-0.1,0.1)node[above left]{$b$};
%equations
\draw (6.5,\h/2) node[right]{$
\begin{aligned}
A_a&=A_c=b w\\
\lambda_1&=N_1 \phi\\
\lambda_2&=N_2 \phi\\
\phi&=\frac{N_1 i_1 + N_2 i_2}{\Re_a+\Re_c}
\end{aligned}
$};
%urdu
\draw (4,\h/2) node[above right]{موٹائی $=b$};
\draw(4,\h/2) node[below right]{گہرائی $=w$};
\end{tikzpicture}%
\caption{دو لچھے والا مقناطیسی دور۔}
\label{شکل_مقناطیسی_ادوار_دو_لچھے_ایک_درز}
\end{figure}
دونوں لچھوں کا مجموعی مقناطیسی دباو، \عددیء{N_1 i_1+N_2 i_2}،  مقناطیسی بہاو \عددیء{\phi} پیدا کرتا ہے۔ اس مقناطیسی بہاو  کا پہلے  لچھے  کے ساتھ  ارتباط
\begin{align}
\lambda_1=N_1 \phi=N_1^2  \frac{\mu_0 A_a}{l_a} i_1 +N_1 N_2  \frac{\mu_0 A_a}{l_a} i_2
\end{align}
یعنی
\begin{align}\label{مساوات_مقناطیسی_دور_ارتباط_دو_لچھے}
\lambda_1 = L_{11} i_1+L_{12} i_2
\end{align}
ہے جہاں \عددی{L_{11}} اور \عددی{L_{12}} سے مراد درج ذیل ہے۔
\begin{align}
L_{11}&=N_1^2  \frac{\mu_0 A_a}{l_a}\\
L_{12}&=N_1 N_2  \frac{\mu_0 A_a}{l_a}
\end{align}
\عددیء{L_{11}} پہلے لچھے کا  \اصطلاح{خود امالہ}\فرہنگ{خود امالہ}\حاشیہب{self inductance}\فرہنگ{self inductance}  ہے اور  \عددیء{L_{11} i_1} اس لچھے کے اپنے برقی رو \عددیء{i_1} سے پیدا مقناطیسی بہاو  کے ساتھ  ارتباط بہاو ہے جسے \اصطلاح{خود ارتباط بہاو}\فرہنگ{خود ارتباط بہاو}\حاشیہب{self flux linkage}\فرہنگ{self flux linkage} کہتے ہیں۔\عددیء{L_{12}}  اِن دونوں لچھوں  کا  \اصطلاح{مشترکہ امالہ}\فرہنگ{مشترکہ امالہ}\حاشیہب{mutual inductance}\فرہنگ{mutual inductance} ہے اور  \عددیء{L_{12} i_2}  لچھا-1  کے ساتھ  \عددیء{i_2} سے پیدا بہاو کے ساتھ  ارتباط بہاو  ہے جسے \اصطلاح{مشترکہ ارتباط بہاو}\فرہنگ{مشترکہ ارتباط امالہ}\حاشیہب{mutual flux linkage}\فرہنگ{mutual flux linkage}  کہتے ہیں ۔ بالکل اسی طرح ہم دوسرے لچھے کے لئے درج ذیل لکھ سکتے ہیں
\begin{align}
\lambda_2&=N_2 \phi=N_2 N_1 \frac{\mu_0 A_a}{l_a} i_1+N_2^2 \frac{\mu_0 A_a}{l_a} i_2  \nonumber \\
&=L_{21} i_1+L_{22} i_2 \label{مساوات_مقناطیسی_دور_دوسرے_لچھے_کی_ارتباط}
\end{align}
جہاں \عددی{L_{22}} اور \عددی{L_{21}} سے مراد درج ذیل ہے۔
\begin{align}
L_{22}&=N_2^2 \frac{\mu_0 A_a}{l_a}\\
L_{21}&=L_{12}=N_2 N_1 \frac{\mu_0 A_a}{l_a} \label{مساوات_مقناطیسی_دور_مشترکہ_امالہ_یکساں}
\end{align}
\عددیء{L_{22}} لچھا-2  کا خود امالہ اور  \عددیء{L_{21}=L_{12}} دونوں لچھوں کا مشترکہ امالہ ہے۔امالہ کا تصور اس وقت کارآمد ہوتا ہے جب مقناطیسی مستقل \عددیء{\mu}  کو اٹل تصور کرنا ممکن ہو۔

مساوات \حوالہ{مساوات_مقناطیسی_دور_خود_امالہ_تعریف}  کو مساوات \حوالہ{مساوات_مقناطیسی_دور_فیراڈے_قانون}  میں پر کرتے ہیں۔
\begin{align}
e=\frac{\partial \lambda}{\partial t}=\frac{\partial \left( L i\right) }{\partial t}
\end{align}
اگر امالہ کی قیمت اٹل ہو، جیسا کہ ساکن مشینوں  میں ہوتا ہے، تب ہمیں  امالہ کی جانی پہچانی مساوات
\begin{align}
e=L \frac{\partial i}{\partial t}
\end{align}
ملتی ہے۔ اگر امالہ بھی تبدیل ہو، جیسا کہ موٹروں اور جنریٹروں میں ہوتا ہے، تب درج ذیل ہو گا۔
\begin{align}
e= L \frac{\partial i}{\partial t} + i \frac{\partial L}{\partial t}
\end{align}

\اصطلاح{توانائی}\فرہنگ{توانائی}\حاشیہب{energy}\فرہنگ{energy}  کی اکائی \اصطلاح{جاول}\فرہنگ{جاول}\حاشیہب{Joule}\فرہنگ{Joule} \عددیء{J}\حاشیہد{جیمس پریسقوٹ جاول انگلستانی سائنسدان جنہوں نے حرارت اور میکانی کام کا رشتہ دریافت کیا} ہے اور \اصطلاح{طاقت}\فرہنگ{طاقت}\حاشیہب{power}\فرہنگ{power}  کی اکائی\حاشیہد{سکاٹلینڈ کے جیمز واٹ جنہوں نے بخارات پر چلنے والے انجن پر کام کیا} جاول فی سیکنڈ ہے جس کو \اصطلاح{واٹ}\فرہنگ{واٹ}\حاشیہب{Watt}\فرہنگ{Watt} \عددیء{W} کا نام دیا گیا  ہے۔

اس کتاب میں توانائی یا کام کو \عددیء{W} سے ظاہر کیا جائے گا اگرچہ طاقت کی اکائی واٹ \عددیء{W} کے لئے بھی یہی علامت استعمال ہوتی ہے۔امید کی جاتی ہے کہ متن سے  اصل مطلب جاننا ممکن ہو گا۔

وقت \عددیء{t} کے ساتھ توانائی \عددیء{W} کی تبدیلی کی شرح کو \اصطلاح{طاقت} \عددیء{p} کہتے ہیں۔یوں درج ذیل لکھا جا سکتا ہے۔
\begin{align}
p=\frac{\dif W}{\dif t} = ie = i \frac{\dif \lambda}{\dif t}
\end{align} 
مقناطیسی دور میں  لمحہ \عددیء{t_1} تا \عددیء{t_2}  مقناطیسی توانائی کی تبدیلی کو تکمل کے ذریعہ حاصل کیا جا سکتا ہے:
\begin{align}
\Delta W = \int_{t1}^{t2} p \dif t =\int_{\lambda1}^{\lambda2} i \dif \lambda
\end{align}
ایک لچھے کا مقناطیسی دور، جس میں  امالہ کی قیمت اٹل ہو، کے لئے درج ذیل لکھا جا سکتا ہے۔
\begin{align}
\Delta W = \int_{\lambda1}^{\lambda2} i \dif \lambda=\int_{\lambda1}^{\lambda2} \frac{\lambda}{L} \dif \lambda=\frac{1}{2 L} \left(\lambda_2^2-\lambda_1^2 \right)
\end{align}

یوں \عددیء{t_1} پر \عددیء{\lambda_1=0} تصور کرتے ہوئے  کسی بھی \عددیء{\lambda} پر مقناطیسی توانائی درج ذیل ہو گی۔
\begin{align}
 W=\frac{\lambda^2}{2L}=\frac{L i^2}{2}
\end{align}

\حصہ{مقناطیسی مادہ کے خواص}\شناخت{حصہ_مقناطیسی_دور_مقناطیسی_مادہ_کے_خصوصیات}
قالب کے استعمال سے دو فوائد حاصل ہوتے ہیں۔ قالب کے استعمال سے کم مقناطیسی دباو، زیادہ مقناطیسی بہاو پیدا کرتا ہے اور مقناطیسی بہاو کو پسند کی راہ پر رہنے کا پابند بنایا جا سکتا ہے۔ یک دوری ٹرانسفارمروں میں قالب کے استعمال سے مقناطیسی بہاو کو اس طرح پابند کیا جاتا ہے کہ تمام لچھوں میں یکساں بہاو پایا جاتا ہو۔ موٹروں میں قالب کے استعمال سے مقناطیسی بہاو کو یوں پابند کیا جاتا ہے کہ زیادہ سے زیادہ قوت پیدا ہو جبکہ جنریٹروں میں زیادہ سے زیادہ برقی دباو حاصل کرنے کی نیت سے بہاو کو پابند کیا جاتا ہے۔
%==========
\begin{figure}
\centering
%\includegraphics{figMagneticCircuitsBrillouinFunctionHysterisisLoop}
\pgfmathsetmacro{\J}{0.5}
\pgfmathsetmacro{\k}{(2*\J+1)/(2*\J)}
\def\kcoth(#1){(e^(#1)+e^(-#1))/(e^(#1)-e^(-#1))}    %this gives correct answers
\def\Brillouin(#1){and(#1>-0.001, #1< 0.001)*(0)+or(#1<-0.001, #1>0.001)*(\k*(e^(\k*#1)+e^(-\k*#1))/(e^(\k*#1)-e^(-\k*#1))-1/(2*\J)*(e^(#1/(2*\J))+e^(-#1/(2*\J)))/(e^(#1/(2*\J))-e^(-#1/(2*\J))))}    %this is correct 
%================
\begin{subfigure}{0.45\textwidth}
\centering
\begin{tikzpicture}
	\begin{axis}[small,  axis lines=middle, axis line style={-},ticks=none,ylabel=$B$, xlabel=$H$,enlargelimits=true,
xlabel style={at={(current axis.right of origin)},anchor=north},
 ylabel style={at={(current axis.above origin)},anchor=east},]
\pgfmathsetmacro{\ka}{0.5}
\pgfmathsetmacro{\kaa}{0.51}
\pgfmathsetmacro{\kpi}{2}
\addplot[ domain=0:\kpi]{\Brillouin(x)}node[pos=0,shift={(-1ex,-1ex)}]{$a$}node[pos=1,above]{$b$};
\addplot[ domain=\kpi:-\kpi]{0.2*cos(x*90/\kpi)^2+\Brillouin(x)}node[pos=0.5,above left,xshift=0.5ex]{$c$}node[pos=0.55,above,xshift=-0.5ex]{$d$}node[pos=1,left]{$e$};
\addplot[domain=-\kpi:0]{-0.15*cos(x*90/\kpi)^2+\Brillouin(x)}node[pos=1,below right,xshift=-0.5ex]{$f$};
\addplot[domain=0:\kpi]{-0.15*cos(x*90/\kpi)^2+\Brillouin(x)-0.02*(-1+e^(x/2))}node[pos=0.1,below right,xshift=-0.5ex]{$g$}node[pos=1,below]{$h$};
%arrows
\node[](a) at (axis cs:\ka,{\Brillouin(\ka)}){};
\node[](b) at (axis cs:\kaa,{\Brillouin(\kaa)}){};
\draw[-stealth](a)--(b);
%
\node[](bb) at (axis cs:\ka,{0.2*cos(\ka*90/\kpi)^2+\Brillouin(\ka)}){};
\node[](cc) at (axis cs:\kaa,{0.2*cos(\kaa*90/\kpi)^2+\Brillouin(\kaa)}){};
\draw[-stealth](cc)--(bb);
%
\node[](c) at (axis cs:-\ka,{0.2*cos(-\ka*90/\kpi)^2+\Brillouin(-\ka)}){};
\node[](d) at (axis cs:-\kaa,{0.2*cos(-\kaa*90/\kpi)^2+\Brillouin(-\kaa)}){};
\draw[-stealth](c)--(d);
%
\node[](ee) at (axis cs:-\ka,{-0.15*cos(-\ka*90/\kpi)^2+\Brillouin(-\ka)}){};
\node[](dd) at (axis cs:-\kaa,{-0.15*cos(-\kaa*90/\kpi)^2+\Brillouin(-\kaa)}){};
\draw[-stealth](dd)--(ee);
%
\node[](f) at (axis cs:\ka,{-0.15*cos(\ka*90/\kpi)^2+\Brillouin(\ka)}){};
\node[](h) at (axis cs:\kaa,{-0.15*cos(\kaa*90/\kpi)^2+\Brillouin(\kaa)}){};
\draw[-stealth](f)--(h);
%\draw(axis cs:0,0)node[below left]{$a$};
\end{axis}
\end{tikzpicture}%
\caption{}
\end{subfigure}%
\begin{subfigure}{0.45\textwidth}
\centering
\begin{tikzpicture}
\begin{axis}[small,  axis lines=middle, axis line style={-},ticks=none,ylabel=$B$, xlabel=$H$,enlargelimits=true,
xlabel style={at={(current axis.right of origin)},anchor=north}, ylabel style={at={(current axis.above origin)},anchor=east},]
\pgfmathsetmacro{\ka}{1}
\pgfmathsetmacro{\kaa}{1.1}
\pgfmathsetmacro{\kb}{-1}
\pgfmathsetmacro{\kbb}{-1.1}
\pgfmathsetmacro{\kpi}{2}
\addplot[ domain=\kpi:-\kpi]{0.2*cos(x*90/\kpi)^2+\Brillouin(x)};
\addplot[domain=-\kpi:\kpi]{-0.2*cos(x*90/\kpi)^2+\Brillouin(x)};
%arrows
\node[](a) at (axis cs:\ka,{0.2*cos(\ka*90/\kpi)^2+\Brillouin(\ka)}){};
\node[](b) at (axis cs:\kaa,{0.2*cos(\kaa*90/\kpi)^2+\Brillouin(\kaa)}){};
\node[](d) at (axis cs:\kb,{-0.2*cos(\kb*90/\kpi)^2+\Brillouin(\kb)}){};
\node[](c) at (axis cs:\kbb,{-0.2*cos(\kbb*90/\kpi)^2+\Brillouin(\kbb)}){};
\draw[-stealth](b)--(a);
\draw[-stealth](c)--(d);
\end{axis}
\end{tikzpicture}
\caption{}
\end{subfigure}
\caption{$B-H$   خطوط یا مقناطیسی چال کے دائرے۔}
\label{شکل_مقناطیسی_چال}
\end{figure}


مقناطیسی مادہ کی \عددیء{B} اور \عددیء{H} کا تعلق  ترسیم کی صورت میں پیش کیا جاتا ہے۔ لوہا نما مقناطیسی مادے کی \عددیء{B-H}  ترسیم شکل \حوالہ{شکل_مقناطیسی_چال}-الف میں دکھائی گئی ہے۔ایک لوہا نما مقناطیسی مادہ جس میں  مقناطیسی اثر نہیں پایا جاتا ہو کو نقطہ \عددیء{a} سے ظاہر کیا گیا ہے۔اس نقطہ پر درج ذیل ہوں گے۔
\begin{gather}
\begin{aligned}
H_a&=0\\
B_a&=0
\end{aligned}
\end{gather}

مقناطیسی مادہ کو لچھے میں رکھ کر اس پر مقناطیسی دباو لاگو کیا جا سکتا ہے۔ مقناطیسی میدان کی شدت \عددیء{H}  لاگو کرنے سے لوہا نما مقناطیسی مادے میں کثافت مقناطیسی بہاو  \عددیء{B} پیدا ہو گا۔میدانی شدت بڑھانے سے کثافت مقناطیسی بہاو بھی بڑھے گا۔ \عددیء{a} سے شروع ہوتا ہوا   تیردار قوس اس عمل کو ظاہر کرتا ہے۔میدانی شدت کو نقطہ \عددیء{b}  تک بڑھایا گیا ہے جہاں  \عددیء{H_b} اور \عددیء{B_b} ہوں گے۔


نقطہ \عددیء{b} تک پہنچنے کے بعد میدانی شدت کم کرتے ہوئے دیکھا گیا ہے کہ واپسی  قوس ایک مختلف راستہ اختیار کرتا ہے۔یوں نقطہ  \عددیء{b} سے میدانی شدت کم کرتے ہوئے صفر کرنے سے  لوہا نما مادہ کی کثافتِ مقناطیسی بہاو کم ہو کر نقطہ \عددیء{c} پر آن پہنچتا ہے۔نقطہ \عددیء{b} سے نقطہ \عددیء{c} تک تیردار قوس اس عمل کو ظاہر کرتا ہے۔نقطہ \عددیء{c} پر بیرونی میدانی شدت صفر ہے لیکن لوہا نما مادے کی کثافتِ مقناطیسی بہاو صفر نہیں ہے۔یہ مادہ ایک مقناطیس بن گیا ہے جس کی کثافتِ مقناطیسی بہاو  \عددیء{B_c} ہے۔اس مقدار کو \اصطلاح{بقایا کثافتِ مقناطیسی بہاو}\فرہنگ{کثافت مقناطیسی بہاو!بقایا}\فرہنگ{magnetic flux!residual}\حاشیہب{residual magnetic flux}  کہتے ہیں۔مصنوعی مقناطیس اسی طرح بنایا جاتا ہے۔

نقطہ \عددی{c} سے میدانی شدت منفی رخ  بڑھانے سے  \عددیء{B} کم ہوتے ہوتے آخر کار ایک مرتبہ دوبارہ صفر ہو جائے گا۔اس نقطہ کو \عددیء{d} سے ظاہر کیا گیا ہے۔مقناطیسیت ختم کرنے کے لئے درکار میدانی شدت کی مقدار  \عددیء{\abs{H_d}} کو مقناطیسیت ختم کرنے والی شدت یا مختصراً \اصطلاح{خاتم شدت}\فرہنگ{مقناطیس!خاتم شدت}\حاشیہب{coercivity}\فرہنگ{coercivity} کہتے ہیں۔

منفی رخ  میدانی شدت مزید بڑھانے سے  نقطہ \عددیء{e} حاصل ہو گا۔ اس کے بعد منفی رخ  کی میدانی شدت کی مطلق قیمت کم کرنے سے  نقطہ \عددیء{f} حاصل ہو گا جہاں میدانی شدت صفر ہونے کے باوجود کثافتِ مقناطیسی بہاو صفر نہیں ہے۔اس نقطہ پر لوہا نما مادہ اُلٹ رخ مقناطیس بن چکا ہے اور \عددیء{B_f} بقایا کثافتِ مقناطیسی بہاو ہے۔اسی طرح اس رخ مقناطیسیت ختم کرنے کی شدت \عددیء{\abs{H_g}} ہے۔میدانی شدت بڑھاتے ہوئے نقطہ \عددی{b} کی بجائے   نقطہ \عددیء{h} حاصل ہو گا۔

برقی شدت کو متواتر اسی طرح پہلے ایک رخ اور پھر مخالف (دوسری) رخ  ایک خاص حد تک پہنچانے سے  آخر کار \عددیء{B-H}  منحنی کا ایک بند دائرہ حاصل ہو گا جسے شکل \حوالہ{شکل_مقناطیسی_چال}-ب میں دکھایا گیا ہے۔اس دائرہ پر خلاف گھڑی سفر ہو گا۔شکل \حوالہ{شکل_مقناطیسی_چال}-ب کو  \اصطلاح{مقناطیسی چال} کا دائرہ\فرہنگ{مقناطیس!چال کا دائرہ}\حاشیہب{hysteresis loop}\فرہنگ{hysteresis loop}  کہتے ہیں۔

\begin{figure}
\centering
%\includegraphics{figMagneticCircuitsM5curve}
\begin{tikzpicture}
\begin{semilogxaxis}[small,axis lines*=middle,xlabel={$H(\si{\ampere t\per\meter})$},ylabel={$B(\si{\tesla})$},xlabel style={at={(current axis.right of origin)},anchor=west},ylabel style={rotate={-90},at={(current axis.above origin)},anchor=south west}]
\addplot [mark=none]coordinates {
(2,0.04)
(3,0.095)
(4,0.16)
(5,0.24)
(6,0.33)
(7,0.44)
(8,0.56)
(9,0.7)
(10,0.835)
(11.22,1)
(12.59,1.1)
(14.96,1.2)
(17.78,1.3)
(20,1.34)
(23.77,1.4)
(30,1.48)
(40,1.54)
(50,1.58)
(60,1.601)
(70,1.626)
(80,1.64)
(90,1.655)
(100,1.662)
(200,1.72)
(300,1.752)
(400,1.78)
(500,1.8)
(600,1.81)
(700,1.824)
(800,1.835)
(900,1.846)
(1000,1.852)
(2000,1.9)
(10000,2)
(40000,2.06)
(70000,2.08)
};
\end{semilogxaxis}
\end{tikzpicture}%
\caption{فولاد $M5$ کی $0.3048$ ملی میٹر موٹی پتری کی ترسیم۔ میدانی شدت کا پیمانہ لاگ ہے۔}
\label{شکل_مقناطیسی_ادوار_ایم_پانچ_پتری_کا_خط}
\end{figure}

مختلف \عددیء{H} کے لئے  شکل \حوالہ{شکل_مقناطیسی_چال}-ب حاصل کر کے ایک ہی کاغذ پر کھینچنے کے بعد ان تمام کے  \عددیء{b} نقطے جوڑنے سے شکل \حوالہ{شکل_مقناطیسی_ادوار_ایم_پانچ_پتری_کا_خط} میں دکھائی گئی \عددیء{B-H} ترسیم حاصل ہو گی۔ ٹرانسفارمروں میں استعمال ہونے والی  \عددیء{0.3048}  ملی میٹر موٹی \عددیء{M5} قالبی  پتری کی \عددیء{B-H} ترسیم شکل \حوالہ{شکل_مقناطیسی_ادوار_ایم_پانچ_پتری_کا_خط} میں  دکھائی گئی ہے۔ اس ترسیم میں موجود مواد جدول \حوالہ{جدول_مقناطیسی_ادوار_کثافت_بہاو_بالمقابل_شدت}  میں بھی دیا گیا ہے۔عموماً مقناطیسی مسائل حل کرتے ہوئے شکل \حوالہ{شکل_مقناطیسی_چال} کی جگہ شکل \حوالہ{شکل_مقناطیسی_ادوار_ایم_پانچ_پتری_کا_خط} طرز  کی ترسیم استعمال کی جاتی ہے۔دھیان رہے کہ اس ترسیم میں \عددیء{H}  کا پیمانہ \اصطلاح{لاگ}\حاشیہب{log}  ہے۔

لوہا نما مقناطیسی مادے  پر لاگو مقناطیسی شدت بڑھانے سے کثافتِ مقناطیسی بہاو بڑھنے کی شرح بتدریج کم ہوتی جاتی ہے حتیٰ کہ آخر کار یہ شرح  خلاء کی شرح  \عددیء{\mu_0} کے برابر ہو جاتی ہے \عددی{(\tfrac{\Delta B}{\Delta H}= \mu_0)}۔
اس اثر کو \اصطلاح{سیرابیت}\فرہنگ{سیرابیت}\حاشیہب{saturation}\فرہنگ{saturation} کہتے ہیں جو  شکل \حوالہ{شکل_مقناطیسی_ادوار_ایم_پانچ_پتری_کا_خط}  میں واضح ہے۔

شکل \حوالہ{شکل_مقناطیسی_چال} سے واضح ہے کہ \عددیء{H} کی کسی بھی قیمت پر \عددیء{B} کی  دو ممکنہ قیمتیں ہوں گی۔ بڑھتے مقناطیسی بہاو کی صورت میں ترسیم میں نیچے سے اُوپر جانے والی منحنی \عددیء{B} اور \عددیء{H} کا تعلق پیش کرے گی جبکہ گھٹتے ہوئے مقناطیسی بہاو کی صورت میں  اوپر سے نیچے جانے والی منحنی اس تعلق کو پیش کرے  گی۔  چونکہ \عددیء{\mu=B/H} ہے  لہٰذا \عددیء{B} کی  مقدار تبدیل ہونے سے \عددیء{\mu} کی قیمت  بھی تبدیل ہو گی۔ باوجود اس کے ہم مقناطیسی ادوار میں  \عددیء{\mu} کو ایک مستقل تصور کرتے ہیں۔ ایسا کرنے سے نتائج پر عموماً زیادہ اثر انداز نہیں ہوتا ہے۔
%
\ابتدا{مثال}
شکل \حوالہ{شکل_مقناطیسی_ادوار_ایم_پانچ_پتری_کا_خط}  یا اس کے مساوی جدول \حوالہ{جدول_مقناطیسی_ادوار_کثافت_بہاو_بالمقابل_شدت} میں دی گئی مواد  استعمال کرتے ہوئے شکل \حوالہ{شکل_مقناطیسی__کثافت_مقناطیسی_بہاو_اور_شدت}  کی خلاء میں ایک ٹسلا اور دو ٹسلا کثافت  مقناطیسی بہاو حاصل کرنے کے لئے درکار برقی رو معلوم کریں۔درج ذیل معلومات استعمال کریں۔ قالب اور خلاء کا رقبہ عمودی تراش ایک دوسرے جتنا لیں۔
\begin{align*}
b=\SI{5}{\centi\meter},w=\SI{4}{\centi\meter},l_a=\SI{3}{\milli\meter},l_c=\SI{30}{\centi\meter},N=1000
\end{align*}


حل:\quad
 ایک ٹسلا کے لئے۔\\
 جدول \حوالہ{جدول_مقناطیسی_ادوار_کثافت_بہاو_بالمقابل_شدت} کے تحت قالب میں \عددیء{1} ٹسلا  کے لئے  قالب کو \عددیء{11.22}  ایمپیئر و چکر فی  میٹر قیمت کی شدت \عددیء{H}  درکار ہو گی۔یوں \عددیء{30} سم لمبے قالب کو \عددیء{0.3\times 11.22=3.366}  ایمپیئر چکر درکار ہوں گے۔

خلاء کو درج ذیل ایمپیئر و چکر فی میٹر شدت درکار ہے۔
\begin{align*}
H=\frac{B}{\mu_0}=\frac{1}{4\pi 10^{-7}}=\num{795775}
\end{align*}
یوں \عددیء{ 3 } ملی میٹر  خلاء کو \عددیء{0.003 \times 795775=2387} ایمپیئر چکر درکار ہوں گے۔ کُل ایمپیئر و چکر ان دونوں کا مجموعہ \عددیء{3.366+2387=2390.366} ہو گا جس سے  درج ذیل حاصل ہوتا ہے۔
\begin{align*}
i=\frac{2390.366}{1000}=\SI{2.39}{\ampere}
\end{align*}	

حل: دو ٹسلا کے لئے۔\\
جدول \حوالہ{جدول_مقناطیسی_ادوار_کثافت_بہاو_بالمقابل_شدت} کے تحت قالب میں \عددیء{2} ٹسلا  کثافت کے لئے  قالب کو \عددیء{10000} ایمپیئر و چکر فی میٹر \عددیء{H} درکار ہو گی۔یوں \عددیء{30} سم  قالب کو \عددیء{0.3 \times 10000=3000} ایمپیئر چکر درکار ہوں گے۔خلاء کو
\begin{align*}
H=\frac{B}{\mu_0}=\frac{2}{4\pi 10^{-7}}=\num{1591549}
\end{align*}
ایمپیئر و چکر فی میٹر درکار ہیں لہٰذا \عددیء{3} ملی میٹر لمبی خلاء کو  \عددیء{0.003 \times 1591549=4775}  ایمپیئر چکر درکار ہوں گے۔یوں کُل ایمپیئر و چکر \عددیء{3000+4775=7775} ہیں جن سے  درج ذیل حاصل کیا جا سکتا ہے۔
\begin{align*}
i=\frac{7775}{1000}=\SI{7.775}{\ampere}
\end{align*}

اس مثال میں مقناطیسی سیرابیت  واضح ہے۔ 
\انتہا{مثال}
%
\begin{table}
\caption{مقناطیسی بہاو بالمقابل شدت}
\label{جدول_مقناطیسی_ادوار_کثافت_بہاو_بالمقابل_شدت}
\begin{otherlanguage}{english}
\begin{tabular}{l l l l   l l l l   l l l l}
$H$&$B$&$H$&$B$&$H$&$B$&$H$&$B$&$H$&$B$&$H$&$B$\\
\midrule
9000&1.998&1000&1.852&           200&1.720 &30&1.480               &9&0.700&  0&0.000    \\
10000&2.000&2000&1.900&         300&1.752 &40&1.540           &10&0.835&  2&0.040    \\
20000&2.020&3000&1.936&         400&1.780 &50&1.580          &11.22&1.000&  3&0.095    \\
30000&2.040& 4000&1.952&        500&1.800 &60&1.601         &12.59&1.100 &  4&0.160    \\
40000&2.048&5000&1.968&         600&1.810 &70&1.626          &14.96&1.200&   5&0.240    \\
50000&2.060&6000&1.975&         700&1.824 &80&1.640         &17.78&1.300&  6&0.330    \\
60000&2.070&7000&1.980&         800&1.835  &90&1.655         &20&1.340&  7&0.440    \\
 70000&2.080&8000&1.985&        900&1.846 &100&1.662          &23.77&1.400& 8&0.560    \\
\bottomrule
\end{tabular}
\end{otherlanguage}
\end{table}
%

\حصہ{ہیجان شدہ لچھا}
بدلتا رو بجلی میں برقی دباو اور مقناطیسی بہاو عموماً سائن نما ہوتے ہیں جن کا وقت کے ساتھ تعلق \عددیء{\sin \omega t} یا \عددیء{\cos \omega t} ہو گا۔ اس حصہ میں بدلتا رو سے لچھا ہیجان کرنا اور اس سے نمودار ہونے والی برقی توانائی  کے ضیاع  پر تذکرہ  کیا جائے گا۔  قالب میں کثافت مقناطیسی بہاو
\begin{align}
B=B_0 \sin \omega t
\end{align}
کی صورت میں قالب میں درج ذیل بدلتا مقناطیسی بہاو \عددیء{\varphi} پیدا ہو گا۔
\begin{align}
\varphi=A_c B=A_c B_0 \sin \omega t=\phi_0 \sin \omega t
\end{align}
اس مساوات میں مقناطیسی بہاو کا حیطہ  \عددیء{\phi_0}، کثافت مقناطیسی بہاو کا حیطہ \عددیء{B_0}، قالب کا رقبہ عمودی تراش  \عددیء{A_c} (جو ہر مقام پر یکساں ہے)، زاویائی تعدد  \عددیء{\omega = 2 \pi f} اور تعدد \عددیء{f}  ہے۔

فیراڈے کے قانون  (مساوات \حوالہ{مساوات_مقناطیسی_دور_فیراڈے_قانون})  کے تحت یہ مقناطیسی بہاو  لچھے میں \عددیء{e(t)} \اصطلاح{امالی برقی دباو}\فرہنگ{امالی!برقی دباو}\حاشیہب{induced voltage}\فرہنگ{induced voltage}  پیدا کرے گا
\begin{gather}
\begin{aligned}
e(t)&=\frac{\partial \lambda}{\partial t}\\
&=\omega N \phi_0 \cos \omega t \\
&=\omega N A_c B_0 \cos \omega t\\
&=E_0 \cos \omega t
\end{aligned}
\end{gather}
جہاں حیطہ \عددی{E_0} درج ذیل ہے۔
\begin{align}
E_0=\omega N \phi_0=2 \pi f N A_c B_0
\end{align}

ہم بدلتے رو مقداروں کے مربع کی اوسط کے جذر  میں دلچسپی رکھتے ہیں جو ان مقداروں کی \اصطلاح{موثر}\فرہنگ{موثر}\حاشیہب{root mean square, rms}\فرہنگ{rms} قیمت ہوتی ہے۔ جیسا صفحہ \حوالہصفحہ{مساوات_بنیادی_سائن_نما_کی_موثر_قیمت} پر مساوات \حوالہ{مساوات_بنیادی_سائن_نما_کی_موثر_قیمت}  میں دیکھا گیا،  سائن نما  موج کی موثر قیمت موج کے حیطہ کی  \عددیء{1/\sqrt{2}} گنّا ہو گی  لہٰذا امالی برقی دباو کی موثر قیمت \عددی{E_{rms}} درج ذیل ہو گی۔
\begin{align}\label{مساوات_مقناطیسی_دور_پیدا_دباو_موثر_قیمت}
E_{rms}=\frac{E_0}{\sqrt{2}}=\frac{2 \pi f N A_c B_0}{\sqrt{2}}=4.44 f N A_c B_0
\end{align}
یہ مساوات بہت اہم ہے  جس کو ہم بار بار استعمال کریں گے۔بدلتے برقی دباو یا بدلتے برقی رو کی قیمت سے مراد ان کی موثر  قیمت ہو گی۔پاکستان میں گھریلو برقی دباو کی موثر قیمت \عددیء{220} وولٹ ہے۔اس سائن نما برقی دباو کی چوٹی \عددی{\sqrt{2} \times 220=311} وولٹ ہو گی۔
%
\ابتدا{مثال}\شناخت{مثال_مقناطیسی_دور_محرک_برقی_رو_کا_گراف}
شکل \حوالہ{شکل_مقناطیسی__سادہ_مقناطیسی_دور_بغیر_درز_دوبارہ} میں لچھے کے \عددیء{27} چکر ہیں۔ قالب کی لمبائی \عددیء{30 } سم جبکہ اس کا رقبہ عمودی تراش \عددیء{229.253} مربع سم ہے۔لچھے  کو گھریلو \عددیء{220} وولٹ موثر برقی دباو سے ہیجان  کیا جاتا ہے۔جدول \حوالہ{جدول_مقناطیسی_ادوار_کثافت_بہاو_بالمقابل_شدت} کی مدد سے مختلف برقی دباو پر محرک برقی رو معلوم کریں اور اس کا خط کھینچیں۔

\begin{figure}
\centering
\begin{tikzpicture}
\def\height{2};
\def\width{1.5};
\def\thick{0.4};
\def\depthX{0.2};
\def\depthY{0.2};
\def\gap{0.05};
%grid
%\draw[gray,thick](0,0) grid (5,3);
%\draw[gray,thin,xstep=0.1,ystep=0.1](0,0) grid (5,3);
%going clockwise from origin
\draw(0,0)--++(0,\height)--++(\width,0)--++(0,-\height)--cycle;
\draw(0,0)++(\thick,\thick)--++(0,\height-2*\thick)--++(\width-2*\thick,0)--++(0,-\height+2*\thick)--cycle;
%
\draw(\thick,\thick)--++(\depthX,\depthY) --++(0,\height-2*\thick-\depthY);
\draw(\thick,\thick)--++(\depthX,\depthY) --++(\width-2*\thick-\depthX,0);
\draw(0,\height)--++(\depthX,\depthY)--++(\width,0)--++(-\depthX,-\depthY);
\draw(\width,0)--++(\depthX,\depthY)--++(0,\height)--++(-\depthX,-\depthY);
%
\draw (0.6,1.4) to [out=45,in=0] (0.2,1.5) to [short,i_<=$i$] (-1,1.5) node[left]{$+$};
\foreach \l in {1.4,1.2,1}{
\draw (0,\l) to [out=-135,in=45] (0.6,\l-0.2);
}
\draw (0,0.8) to (-1,0.8)node[left]{$-$};
%turns
\node at (0,1.15)[left]{$\tau=N i$};
\end{tikzpicture}%
\caption{سادہ مقناطیسی دور (مثال \حوالہ{مثال_مقناطیسی_دور_محرک_برقی_رو_کا_گراف})۔}
\label{شکل_مقناطیسی__سادہ_مقناطیسی_دور_بغیر_درز_دوبارہ}
\end{figure}
حل:\quad
گھریلو برقی دباو \عددیء{50} ہرٹز کی سائن نما موج ہو گی۔
\begin{align}
v=\sqrt{2} \times 220 \cos (2 \pi  50 t)
\end{align}
مساوات \حوالہ{مساوات_مقناطیسی_دور_پیدا_دباو_موثر_قیمت}  کی مدد سے ہم کثافتِ مقناطیسی بہاو کی چوٹی حاصل کرتے ہیں۔
\begin{align}
B_0=\frac{220}{4.44 \times 50 \times 27 \times 0.0229253}=\SI{1.601}{\tesla}
\end{align}
یوں قالب میں کثافتِ مقناطیسی بہاو کا حیطہ  \عددیء{1.601}  ہو گا اور   قالب میں کثافتِ مقناطیسی بہاو کی مساوات درج ذیل ہو گی۔
\begin{align}\label{مساوات_مقناطیسی_دور_سائن_نما_کثافت_بہاو}
B=1.601 \sin \omega t
\end{align}
ہم جدول کی مدد سے   \عددی{0} اور \عددیء{1.601} ٹسلا کے بیچ  مختلف قیمتوں پر درکار محرک برقی رو \عددیء{i_{\phi}} معلوم کرنا چاہتے ہیں۔ہم مختلف \عددیء{B} پر جدول \حوالہ{جدول_مقناطیسی_ادوار_کثافت_بہاو_بالمقابل_شدت} سے قالب کی \عددیء{H} حاصل کریں گے جو  ایک میٹر لمبی قالب کے لئے درکار ایمپیئر و چکر ہوں گے۔اس سے \عددیء{30} سم لمبی قالب کے لئے درکار ایمپیئر و چکر  دریافت کر کے برقی رو حاصل کریں گے۔

%
\begin{table}
\caption{محرک برقی رو}
\label{جدول_مقناطیسی_ادوار_محرک_برقی_رو_بالمقابل_کثافت_بہاو}
\begin{otherlanguage}{english}
\begin{tabular}{l l l l l | l l l l l}
$i_{\varphi}=\frac{0.3 H}{27}$&$0.3H$&$H$&$B$&$\omega t$&$i_{\varphi}=\frac{0.3 H}{27}$&$0.3H$&$H$&$B$&$\omega t$\\
\midrule
0.000&0.000&0&0.000&0.000&0.125&3.366&11.22&1.000&0.675\\
0.022&0.600&2&0.040&0.025&0.140&3.777&12.59&1.100&0.757\\
0.033&0.900&3&0.095&0.059&0.166&4.488&14.96&1.200&0.847\\
0.044&1.200&4&0.160&0.100&0.198&5.334&17.78&1.300&0.948\\
0.056&1.500&5&0.240&0.150&0.222&6.000&20&1.340&0.992\\
0.067&1.800&6&0.330&0.208&0.264&7.131&23.77&1.400&1.064\\
0.078&2.100&7&0.440&0.278&0.333&9.000&30&1.480&1.180\\
0.089&2.400&8&0.560&0.357&0.444&12.000&40&1.540&1.294\\
0.100&2.700&9&0.700&0.453&0.556&15.000&50&1.580&1.409\\
0.111&3.000&10&0.835&0.549&0.667&18.000&60&1.601&1.571\\
\bottomrule
\end{tabular}
\end{otherlanguage}
\end{table}
%
\begin{figure}
\centering
\includegraphics{figExcitationCurrentFromBHbrillouinCurveNeglectingHysterisisA}
\caption{$M5$ پتری کے قالب میں $1.6$ ٹسلا تک ہیجان پیدا کرنے کے لئے درکار ہیجان انگیز برقی رو۔}
\label{شکل_مقناطیسی_ادوار_ہیجان_رو_چال_نظرانداز}
\end{figure}

جدول \حوالہ{جدول_مقناطیسی_ادوار_محرک_برقی_رو_بالمقابل_کثافت_بہاو}  مختلف کثافتِ مقناطیسی بہاو کے لئے درکار محرک برقی رو دیتی ہے۔جدول میں  ہر \عددیء{B} کی قیمت پر  \عددیء{t} کو مساوات \حوالہ{مساوات_مقناطیسی_دور_سائن_نما_کثافت_بہاو}   سے حاصل کیا گیا ہے۔محرک برقی رو بالمقابل \عددیء{t} کا خط شکل \حوالہ{شکل_مقناطیسی_ادوار_ہیجان_رو_چال_نظرانداز} میں دیا گیا ہے۔
\انتہا{مثال}
%
برقی لچھے میں برقی دباو سے ہیجان پیدا کیا جاتا ہے۔ہیجان شدہ لچھا میں گزرتے برقی رو \عددیء{i_{\varphi}} کی بنا  قالب میں مقناطیسی بہاو پیدا ہو گا۔ اس برقی رو \عددیء{i_{\varphi}} کو \اصطلاح{ہیجان انگیز برقی رو}\فرہنگ{برقی رو!ہیجان انگیز}\حاشیہب{excitation current}\فرہنگ{excitation current}  کہتے ہیں۔

مثال \حوالہ{مثال_مقناطیسی_دور_محرک_برقی_رو_کا_گراف} میں ہیجان انگیز برقی رو معلوم کی گئی جسے شکل \حوالہ{شکل_مقناطیسی_ادوار_ہیجان_رو_چال_نظرانداز} میں دکھایا گیا۔اسے حاصل کرتے وقت \اصطلاح{مقناطیسی چال}\فرہنگ{مقناطیسی چال}\حاشیہب{hysteresis} کو نظر انداز کیا گیا۔شکل \حوالہ{شکل_مقناطیسی_ادوار_ہیجان_رو_بشمول_اثر_چال} میں ہیجان انگیز برقی رو \عددیء{i_\varphi} دکھائی گئی ہے جو مقناطیسی چال کو مدِ نظر رکھ کر حاصل کی گئی ہے۔ اس کو سمجھنا ضروری ہے۔
\begin{figure}
\centering
\includegraphics{figExcitationCurrentFromBHbrillouinCurveA}
\caption{ہیجان انگیز برقی رو۔}
\label{شکل_مقناطیسی_ادوار_ہیجان_رو_بشمول_اثر_چال}
\end{figure}
%
\begin{figure}
%\includegraphics{figMagneticCircuitsVoltAmperePerKgVersusFluxDensity}
\pgfmathsetmacro{\radX}{1.8}
 \pgfmathsetmacro{\radY}{0.4}
\pgfmathsetmacro{\thetaStart}{0} 
\pgfmathsetmacro{\thetaEnd}{180}
\begin{tikzpicture}[yscale=0.75]
%grid
%\draw[gray,thick] (-4*\radX,-4*\radX) grid (4*\radX,4*\radX);
%\draw[gray,thin,xstep=0.1,ystep=0.1] (-4*\radX,-4*\radX) grid (4*\radX,4*\radX);
%axis
\begin{semilogxaxis}[x=2cm,
    log basis x=10,
    log ticks with fixed point,
ymax=1.8,
xmax=100,
axis x line=bottom,
axis y line=left,
axis line style=-,
xlabel={$P_{a,  \textup{موثر}}\,  (\si{\volt\ampere\per\kilogram})$},
ylabel={$B_{\textup{چوٹی}}\, (\si{\tesla})$},
    ]
%{(-3,0) (-2,0.09) (-1,0.34) (0,1.3) (1,1.8) (1.7,1.9)}
  \addplot [smooth]coordinates  {(0.001,0) (0.01,0.075) (0.1,0.283) (1,1.083) (10,1.5) (50,1.583)};
  \end{semilogxaxis}
%
% {(0.001,0) (0.01,0.09) (0.1,0.34) (1,1.3) (10,1.8) (50,1.9)} the flux is multiplied by 50/60 to get data for 50 Hz
\end{tikzpicture}%
\caption{پچاس ہرٹز پر \عددیء{0.3} ملی میٹر موٹی پتری کے لئے درکار موثر وولٹ و  ایمپیئر فی کلوگرام قالب}
\label{شکل_مقناطیسی_دور_درکار_ہیجان_وولٹ_ایمپیئر}
\end{figure}

شکل \حوالہ{شکل_مقناطیسی_ادوار_ہیجان_رو_بشمول_اثر_چال}-الف میں  مقناطیسی چال کا دائرہ دکھایا گیا  ہے۔درج ذیل تعلقات کی بنا مقناطیسی چال کے  خط کو \عددیء{\varphi-i_{\varphi}} کا خط لکھا جا سکتا ہے۔
\begin{gather}
\begin{aligned}
H l& =N i\\
\varphi&=B A_c
\end{aligned}
\end{gather}
قالب میں سائن نما مقناطیسی بہاو \عددیء{\varphi}  کو شکل \حوالہ{شکل_مقناطیسی_ادوار_ہیجان_رو_بشمول_اثر_چال}-ب  میں دکھایا گیا ہے۔سائن نما مقناطیسی بہاو وقت کے ساتھ تبدیل ہوتا ہے۔لمحہ \عددیء{t_1} پر اس کی قیمت  \عددیء{\varphi_1} ہو گی۔مقناطیسی بہاو \عددیء{\varphi_1} حاصل کرنے کے لئے درکار ہیجان انگیز برقی رو \عددیء{i_1} شکل-الف سے حاصل کی جا سکتی ہے۔اسی  ہیجان انگیز برقی رو کو شکل-ب میں  لمحہ \عددیء{t_1} پر دکھایا گیا ہے۔

دھیان رہے کہ لمحہ \عددیء{t_1} پر مقناطیسی بہاو بڑھ رہا ہے لہٰذا مقناطیسی چال کے خط کا درست حصہ استعمال کرنا ضروری ہے۔شکل \حوالہ{شکل_مقناطیسی_ادوار_ہیجان_رو_بشمول_اثر_چال}-الف میں  \عددیء{\varphi-i_{\varphi}}  کے خط میں گھڑی کی سوئیوں کے مخالف رخ گھومتے ہوئے یوں نیچے سے اوپر جاتا ہوا حصہ استعمال کیا گیا ہے۔شکل \حوالہ{شکل_مقناطیسی_چال}-ب میں  تیر کے نشان  مقناطیسی بہاو بڑھنے (نیچے سے اوپر) اور گھٹنے (اوپر سے نیچے) والے حصوں کی نشاندہی کرتے ہیں۔


 لمحہ \عددیء{t_2} پر مقناطیسی بہاو گھٹ رہا ہے۔اس لمحہ پر مقناطیسی بہاو \عددیء{\varphi_2} ہے اور اسے حاصل کرنے کے لئے درکار ہیجان انگیز برقی رو \عددیء{i_2} ہے۔

اسی طرح مختلف لمحات پر درکار ہیجان انگیز برقی رو حاصل کرنے سے شکل \حوالہ{شکل_مقناطیسی_ادوار_ہیجان_رو_بشمول_اثر_چال}-ب کا    \عددیء{i_{\varphi}}  خط ملتا ہے جو  غیر سائن نما ہے۔

آپ جانتے ہیں کہ  \عددیء{\varphi=\phi_0 \sin \omega t} کی صورت میں برقی دباو \عددیء{e=N \tfrac{\dif \varphi}{\dif t}=N \phi_0 \omega \cos \omega t} ہو گا۔شکل \حوالہ{شکل_مقناطیسی_ادوار_ہیجان_رو_بشمول_اثر_چال}-ب میں اس برقی دباو کو بھی دکھایا گیا ہے۔آپ دیکھ سکتے ہیں کہ برقی دباو سے  مقناطیسی بہاو \عددیء{90 ^{\circ}} تاخیر سے ہے۔

قالب میں  \عددیء{B=B_0 \sin \omega t} کی صورت میں  \عددیء{H} اور \عددیء{i_{\varphi}}  غیر سائن نما ہوں گے جن  کی موثر قیمتوں \عددیء{H_{c,rms}} اور  \عددیء{i_{\varphi,rms}} کا تعلق درج ذیل ہو گا۔
\begin{align}\label{مساوات_مقناطیسی_دور_دباو_برابر_شدت_ضرب_لمبائی}
N i_{\varphi,rms}=l_c H_{c,rms}
\end{align}
مساوات \حوالہ{مساوات_مقناطیسی_دور_پیدا_دباو_موثر_قیمت}   اور مساوات \حوالہ{مساوات_مقناطیسی_دور_دباو_برابر_شدت_ضرب_لمبائی}  سے درج ذیل حاصل ہو گا
\begin{align}\label{مساوات_مقناطیسی_دور_درکار_دباو_ضرب_رو}
E_{rms} i_{\varphi,rms}=\sqrt{2} \pi f B_0 H_{c,rms} A_c l_c
\end{align}
جہاں  \عددیء{A_c l_c} قالب کا حجم ہے۔ یوں   \عددیء{A_c l_c} حجم کے قالب  میں  \عددیء{B_0} کثافتِ مقناطیسی بہاو پیدا کرنے کے لئے درکار \عددیء{E_{rms} i_{\varphi,rms}} مساوات \حوالہ{مساوات_مقناطیسی_دور_درکار_دباو_ضرب_رو}  دے گی۔ ایک مقناطیسی قالب جس کا حجم  \عددیء{A_c l_c} اور  میکانی کثافت  \عددیء{\rho_c} ہو، کی کمیت \عددیء{m_c=\rho_c A_c l_c} ہو گی لہٰذا  ایک کلوگرام  قالب کے لئے مساوات \حوالہ{مساوات_مقناطیسی_دور_درکار_دباو_ضرب_رو} کو  درج ذیل روپ میں لکھا جا سکتا ہے۔
\begin{align}
P_a=\frac{E_{rms} i_{\varphi,rms}}{m_c}=\frac{\sqrt{2} \pi f}{\rho_c} B_0 H_{c,rms}
\end{align}
دیکھا جائے تو کسی ایک تعدد  \عددیء{f} پر \عددیء{P_a} کی قیمت صرف قالب پر اور قالب میں \عددیء{B_0} یعنی \عددیء{B_{\textup{چوٹی}}} پر منحصر ہے، چونکہ \عددیء{H_{c,rms}} خود \عددیء{B_0} پر منحصر ہے۔ یہی وجہ ہے کہ  قالب بنانے والے اکائی کمیت کے قالب میں مختلف \عددیء{B_{\textup{چوٹی}}} پیدا کرنے کے لئے درکار \عددیء{E_{rms} i_{\varphi,rms}} کی \عددیء{B_0} بالمقابل \عددیء{P_a} ترسیم مہیا کرتے ہیں۔قالب کی \عددیء{0.3} ملی میٹر موٹی پتری کے لئے ایسی ترسیم  شکل \حوالہ{شکل_مقناطیسی_دور_درکار_ہیجان_وولٹ_ایمپیئر} میں دکھائی گئی ہے۔
