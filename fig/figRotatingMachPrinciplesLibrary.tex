%STATOR
\pgfmathsetmacro{\sRi}{1.25}   %stator inner radius
\pgfmathsetmacro{\sRo}{1.85}  %stator outer radius
%SLOT
\pgfmathsetmacro{\delSlotTheta}{10}   %slot width in degrees
\pgfmathsetmacro{\delR}{0.25}     %slot radial depth
\pgfmathsetmacro{\gap}{0.1}
%==============================
%ROTOR 
\pgfmathsetmacro{\rR}{\sRi-\gap}   %rotor radius  
\pgfmathsetmacro{\pTheta}{50}  %pole face in degrees 
\pgfmathsetmacro{\pX}{0.1}   %poles edge x-direction
\pgfmathsetmacro{\pY}{0.1}    %pole edge  y-direction
%
%\pgfmathsetmacro{\rReduction}{\rT/(2*tan(180/\P))}   %reduction in rotor length due to adjacent rotor
%\pgfmathsetmacro{\rWr}{\rWf-\rReduction-\pX}    %rotor's flat section's effective length
%========
%===============================
%STATOR
%\stator{#slots}{#tilt angle} draws stator with #SLOTS  slots tilted at #tilt angle.
\newcommand{\stator}[2]{
\draw(0,0) circle (\sRo);
\foreach \slot in {1,...,#1}{
\pgfmathsetmacro{\thetaS}{360*(\slot-1)/#1+#2}
\pgfmathsetmacro{\thetaE}{360*\slot/#1+#2}
\draw([shift={(\thetaS+\delSlotTheta/2:\sRi)}]0,0) arc (\thetaS+\delSlotTheta/2:\thetaE-\delSlotTheta/2:\sRi);
\draw(\thetaS-\delSlotTheta/2:\sRi)--++(\thetaS-\delSlotTheta/2:\delR) arc (\thetaS-\delSlotTheta/2:\thetaS+\delSlotTheta/2:\sRi+\delR)--(\thetaS+\delSlotTheta/2:\sRi); } }
%===============================
%===============================
%STATOR EACH SLOT
%\stator{comma separated thetaS/thetaE}  % thetaS/thetaE are two consecutive slots
\newcommand{\statorEach}[1]{
\draw(0,0) circle (\sRo);
\foreach \thetaS/\thetaE in {#1}{
%\pgfmathsetmacro{\thetaS}{360*(\slot-1)/#1+#2}
%\pgfmathsetmacro{\thetaE}{360*\slot/#1+#2}
\draw([shift={(\thetaS+\delSlotTheta/2:\sRi)}]0,0) arc (\thetaS+\delSlotTheta/2:\thetaE-\delSlotTheta/2:\sRi);
\draw(\thetaS-\delSlotTheta/2:\sRi)--++(\thetaS-\delSlotTheta/2:\delR) arc (\thetaS-\delSlotTheta/2:\thetaS+\delSlotTheta/2:\sRi+\delR)--(\thetaS+\delSlotTheta/2:\sRi); } }
%===============================
%ROTOR
%\rotor{#poles}{#tilt angle}   
\newcommand{\rotor}[2]{
\pgfmathsetmacro{\rWf}{\rR*cos(\pTheta/2)}    %rotor  full section width excluding the curved section
\pgfmathsetmacro{\rT}{2*\rR*sin(\pTheta/2)-2*\pY}    %rotor body thickness
%
\pgfmathsetmacro{\rReduction}{\rT*cos(180/#1)/(2*sin(180/#1))}   %reduction in rotor length due to adjacent rotor
\pgfmathsetmacro{\rWr}{\rWf-\rReduction-\pX}    %rotor's flat section's effective length
\foreach \pole in {1,...,#1}{
\pgfmathsetmacro{\angle}{360*(\pole-1)/#1+#2}
\begin{scope}[rotate=\angle]
\draw(\rReduction,-\rT/2)--++(\rWr,0)--++(0,-\pY)--++(\pX,0)  arc (-\pTheta/2:\pTheta/2:\rR)--++(-\pX,0)--++(0,-\pY)--++(-\rWr,0);
\end{scope}} }
%=============================
%==============================
%DOTS and CROSSes around POLES
%\dotCross{#poles}{#tilt angle} puts these around the poles present at the angles
\newcommand{\dotCross}[2]{
\foreach \pole in {1,...,#1}{
\pgfmathsetmacro{\angle}{360*(\pole-1)/#1+#2}
\begin{scope}[rotate=\angle]{
\draw (\rWf-\pX-1.3*\pX,\rT/2+1.3*\pX) circle (2.5pt);
\draw[fill] (\rWf-\pX-1.3*\pX,\rT/2+1.3*\pX) circle (1.5pt);
\draw (\rWf-\pX-1.3*\pX,-\rT/2-1.3*\pX) circle (2.5pt);
\draw (\rWf-\pX-1.3*\pX,-\rT/2-1.3*\pX)++(45:2.2pt)--++(-135:4.4pt);
\draw (\rWf-\pX-1.3*\pX,-\rT/2-1.3*\pX)++(-45:2.2pt)--++(135:4.4pt);  }
\end{scope}} }
%====================================
%DOTS and CROSSes mentioning each Pole
%\dotCrossEachPole{comma separated poles}
\newcommand{\dotCrossEachPole}[1]{
\foreach \angle in {#1}{
\begin{scope}[rotate=\angle]{
\draw (\rWf-\pX-1.3*\pX,\rT/2+1.3*\pX) circle (2.5pt);
\draw[fill] (\rWf-\pX-1.3*\pX,\rT/2+1.3*\pX) circle (1.5pt);
\draw (\rWf-\pX-1.3*\pX,-\rT/2-1.3*\pX) circle (2.5pt);
\draw (\rWf-\pX-1.3*\pX,-\rT/2-1.3*\pX)++(45:2.2pt)--++(-135:4.4pt);
\draw (\rWf-\pX-1.3*\pX,-\rT/2-1.3*\pX)++(-45:2.2pt)--++(135:4.4pt);  }
\end{scope}} }
%=====================================
%DOTS and CROSSes around POLES
%\crossDot{#poles}{#tilt angle} puts these around the poles present at the angles
\newcommand{\crossDot}[2]{
\foreach \pole in {1,...,#1}{
\pgfmathsetmacro{\angle}{360*(\pole-1)/#1+#2}
\begin{scope}[rotate=\angle]{
\draw (\rWf-\pX-1.3*\pX,-\rT/2-1.3*\pX) circle (2.5pt);
\draw[fill] (\rWf-\pX-1.3*\pX,-\rT/2-1.3*\pX) circle (1.5pt);
\draw (\rWf-\pX-1.3*\pX,\rT/2+1.3*\pX) circle (2.5pt);
\draw (\rWf-\pX-1.3*\pX,\rT/2+1.3*\pX)++(45:2.2pt)--++(-135:4.4pt);
\draw (\rWf-\pX-1.3*\pX,\rT/2+1.3*\pX)++(-45:2.2pt)--++(135:4.4pt);  }
\end{scope} } }
%===========================================
%=====================================
%DOTS and CROSSes around POLES
%\crossDotEachPole{comma separated poles}
\newcommand{\crossDotEachPole}[1]{
\foreach \angle in {#1}{
\begin{scope}[rotate=\angle]{
\draw (\rWf-\pX-1.3*\pX,-\rT/2-1.3*\pX) circle (2.5pt);
\draw[fill] (\rWf-\pX-1.3*\pX,-\rT/2-1.3*\pX) circle (1.5pt);
\draw (\rWf-\pX-1.3*\pX,\rT/2+1.3*\pX) circle (2.5pt);
\draw (\rWf-\pX-1.3*\pX,\rT/2+1.3*\pX)++(45:2.2pt)--++(-135:4.4pt);
\draw (\rWf-\pX-1.3*\pX,\rT/2+1.3*\pX)++(-45:2.2pt)--++(135:4.4pt);  }
\end{scope} } }
%===========================================
%SLOT wire as empty circle
%\slotEmptyCircle{comma separated slot angles}
\newcommand{\slotEmptyCircle}[1]{
\foreach \slotTheta in {#1}{
\draw (\slotTheta:\sRi+\delR/2) circle (2.5pt); } }
%==================
%SLOT DOT
%\slotDot{comma separated angles}
\newcommand{\slotDot}[1]{
\foreach \slotTheta in {#1}{
\draw (\slotTheta:\sRi+\delR/2) circle (2.5pt);
\draw[fill] (\slotTheta:\sRi+\delR/2) circle (1.5pt); } }
%===============================
%SLOT CROSS
%\slotCross{comma separated angles}
\newcommand{\slotCross}[1]{
\foreach \slotTheta in {#1}{
\draw (\slotTheta:\sRi+\delR/2) circle (2.5pt);
\draw (\slotTheta:\sRi+\delR/2)++(45:2.2pt)--++(-135:4.4pt);
\draw (\slotTheta:\sRi+\delR/2)++(-45:2.2pt)--++(135:4.4pt); } }
%====================================
%SLOT NAME
%\slotName{comma separated angles/name/NameAtleftORrightOfSlot}
\newcommand{\slotName}[1]{
\foreach \slotTheta/\slName/\position in {#1}{
\draw (\slotTheta:\sRi+2*\delR/3)node[\position]{$\slName$}; } }
%====================================
%====================================
%====================================
%LINEAR FIGURES
%DOT LINEAR   used with CROSS linear, SLOT segment and SlotTop. it doesnot take angles as input
%\dotLinear{comma separated x-locations}     %draws the stator as a linear stator.
\newcommand{\dotLinear}[1]{
\pgfmathsetmacro{\delSlotX}{3.5}
\foreach \x in {#1}{
\draw(\x cm-\delSlotX pt,-\delSlotX pt)--++(0,2*\delSlotX pt)--++(2*\delSlotX pt,0)--++(0,-2*\delSlotX pt);
\draw (\x,0) circle (2.5pt);
\draw[fill] (\x,0) circle (1.5pt); }}
%====================================
%CROSS LINEAR
%\crossLinear{comma separated x-locations}
\newcommand{\crossLinear}[1]{
\pgfmathsetmacro{\delSlotX}{3.5}
\foreach \x in {#1}{
\draw(\x cm-\delSlotX pt,-\delSlotX pt)--++(0,2*\delSlotX pt)--++(2*\delSlotX pt,0)--++(0,-2*\delSlotX pt);
\draw (\x,0) circle (2.5pt);
\draw (\x,0)++(45:2.2pt)--++(-135:4.4pt);
\draw (\x,0)++(-45:2.2pt)--++(135:4.4pt);  }}
%==================================
%====================================
%SLOT SEGMENT
%\slotSegment{comma separated xStart/xEnd locations}
\newcommand{\slotSegment}[1]{
\pgfmathsetmacro{\delSlotX}{3.5}
\foreach \xStart/\xEnd in {#1}{
\draw(\xStart cm+\delSlotX pt,-\delSlotX pt) --(\xEnd cm -\delSlotX pt,-\delSlotX pt);
 }}
%==================================
%this is used after \slotSegment to draw the top cover
\newcommand{\slotTop}[1]{
\foreach \xStart/\xEnd in {#1}{
\draw (\xStart cm-\delSlotX pt,-\delSlotX pt)--++(-0.5 cm,0)--++(0,0.5)--++(\xEnd cm-\xStart cm+1 cm +\delSlotX pt,0)--++(0,-0.5)--(\xEnd cm+\delSlotX pt,-\delSlotX pt);
}}
%================================
%MMF jmps
%\mmf{comma separated xstart/ystart/xend/yend}
\newcommand{\mmf}[1]{
\foreach \xstart/\ystart/\xend/\yend in {#1}{
\draw(\xstart/90,\ystart)--(\xend/90,\ystart)--(\xend/90,\yend);} }
%=================
%===============================
%COMMUTATOR
%\commutator{#slots}{#tilt angle} draws commutator with #SLOTS  slots tilted at #tilt angle.
\newcommand{\commutator}[2]{
%\draw(0,0) circle (\sRo);
\foreach \slot in {1,...,#1}{
\pgfmathsetmacro{\thetaS}{360*(\slot-1)/#1+#2}
\pgfmathsetmacro{\thetaE}{360*\slot/#1+#2}
\draw([shift={(\thetaS+\delSlotTheta/2:\rR)}]0,0) arc (\thetaS+\delSlotTheta/2:\thetaE-\delSlotTheta/2:\rR);
\draw(\thetaS-\delSlotTheta/2:\rR)--++(\thetaS-\delSlotTheta/2:-\delR) arc (\thetaS-\delSlotTheta/2:\thetaS+\delSlotTheta/2:\rR-\delR)--++(\thetaS+\delSlotTheta/2:\delR); } }
%===============================
%===============================
%\draw (0,0)coordinate(k);
%\mymotor{k}{tilt angle}{motor or generator}
\newcommand{\mymotor}[3] % #1 = coordinate name of centre , #2 = tilt
{\draw[thick,rotate=#2] (#1) circle (10pt)
 node[]{\small{#3}} 
++(-12pt,3pt)--++(0,-6pt) --++(2.5pt,0) ++(-2.8pt,6pt)-- ++(2.5pt,0pt);
\draw[thick,rotate=#2] (#1) ++(12pt,3pt)--++(0,-6pt) --++(-2.5pt,0) 
++(2.8pt,6pt)-- ++(-2.5pt,0pt);  }
