\documentclass[b5paper]{standalone}
\usepackage{fontspec}
\usepackage{polyglossia}
\usepackage{circuitikz}
\usepackage{ifthen}   
\usepackage{amsmath}
\usepackage{pgfplots}

\pgfplotsset{compat=1.9}
\usetikzlibrary{decorations.markings}
\usetikzlibrary{calc}
%
\setmainlanguage{english}
\setotherlanguages{arabic}
\newfontfamily\arabicfont[Scale=1.0,Script=Arabic]{Scheherazade}
\newfontfamily\urdufont[Scale=1.0,Script=Arabic]{XB Tabriz}

  
\begin{document}
\begin{urdufont}
\pgfmathsetmacro{\J}{0.5}
\pgfmathsetmacro{\k}{(2*\J+1)/(2*\J)}
%\def\kcoth(#1){(e^(#1)+e^(-#1))/(e^(#1)-e^(-#1))}    %this gives correct answers
%\def\Brillouin(#1){and(#1>-0.001, #1< 0.001)*(0)+or(#1<-0.001, #1>0.001)*(\k*(e^(\k*#1)+e^(-\k*#1))/(e^(\k*#1)-e^(-\k*#1))-1/(2*\J)*(e^(#1/(2*\J))+e^(-#1/(2*\J)))/(e^(#1/(2*\J))-e^(-#1/(2*\J))))}    %this gives correct answers
%
\pgfmathdeclarefunction{kcoth}{1}{\pgfmathparse{(e^(#1)+e^(-#1))/(e^(#1)-e^(-#1))}}%
\pgfmathdeclarefunction{Brillouin}{1}{\pgfmathparse{and(#1>-0.001, #1< 0.001)*(0)+or(#1<-0.001, #1>0.001)*(\k*kcoth(#1*\k)-1/(2*\J)*kcoth(#1/(2*\J)))}}%
%
\pgfmathsetmacro{\kpi}{2}
\pgfmathsetmacro{\kPeak}{Brillouin(\kpi)}
%
\pgfmathdeclarefunction{BrillouinRising}{1}{\pgfmathparse{-0.2*cos(#1*90/\kpi)^2+Brillouin(#1)}}%
\pgfmathdeclarefunction{BrillouinFalling}{1}{\pgfmathparse{0.2*cos(#1*90/\kpi)^2+Brillouin(#1)}}%
%

\pgfmathdeclarefunction{ktimeRising}{1}{\pgfmathparse{asin(BrillouinRising(#1)/\kPeak)}}%
\pgfmathdeclarefunction{ktimeFalling}{1}{\pgfmathparse{asin(BrillouinFalling(#1)/\kPeak)}}%
%
\tikzset{
  set arrow inside/.code={\pgfqkeys{/tikz/arrow inside}{#1}},
  set arrow inside={end/.initial=>, opt/.initial=},
  /pgf/decoration/Mark/.style={
    mark/.expanded=at position #1 with
    {
      \noexpand\arrow[\pgfkeysvalueof{/tikz/arrow inside/opt}]{\pgfkeysvalueof{/tikz/arrow inside/end}}
    }
  },
  arrow inside/.style 2 args={
    set arrow inside={#1},
    postaction={
      decorate,decoration={
        markings,Mark/.list={#2}
      }
    }
  },
}
%============
\begin{tikzpicture}
	\begin{axis}[
  axis x line=middle, 
axis y line=none,
axis line style={-},
 ymin=-1.45*\kpi, ymax=1.45*\kpi,
ticks=none,
%ylabel=$i_\varphi$,
xmin=-200,xmax=200, 
xlabel=$t$,
xlabel style={right},
 ylabel style={above left},
]
\addplot [domain=-180:180,smooth]{\kpi*sin(x)}node [pos=0.4,fill=white] {$\varphi$};
\addplot [domain=-180:180,smooth]{1.4*\kpi*cos(x)}node [pos=0.3,fill=white]{$e$};
%
\addplot[ domain=\kpi:-\kpi, smooth,variable=\x]({-180-ktimeFalling(x)},{0.75*x})coordinate[pos=1](ka);
\addplot[ domain=-\kpi:\kpi, smooth,variable=\x]({ktimeRising(x)},{0.75*x})coordinate[pos=0](kb)coordinate[pos=1](kc)node[pos=0.2,fill=white]{$i_\varphi$};
\addplot[ domain=\kpi:-\kpi, smooth,variable=\x]({180-ktimeFalling(x)},{0.75*x})coordinate[pos=0](kd);
%\addplot[ domain=-\kpi:\kpi,smooth,variable=\x]({360+ktimeRising(x)},{0.5*x})coordinate[pos=0](kd);

\draw (ka) to [out=-45,in=-135] (kb);   %filling empty gaps
\draw (kc) to [out=45,in=135] (kd);
%
\end{axis}
%
%grid
%\draw [gray] (0,0) grid (8,8);
%
\end{tikzpicture}%
%==============================
%================
\begin{tikzpicture}
	\begin{axis}[xscale=0.56,
  axis x line=middle, 
axis y line=middle,
axis line style={-},
 ymin=-1.45*\kPeak, ymax=1.45*\kPeak,
ticks=none,
ylabel=$\varphi$,
  %xmin=-5, xmax=5, xtick={-5,...,5}, 
xlabel=$i_\varphi$,
xlabel style={right},
 ylabel style={left},
]
\addplot[domain=-\kpi:\kpi]{BrillouinRising(x)}[arrow inside={}{0.25,0.75}];
\addplot[ domain=\kpi:-\kpi]{BrillouinFalling(x)}[arrow inside={}{0.25,0.75}];
%
%\addplot[domain=-\kpi:\kpi]{Brillouin(x)};
%
\end{axis}
%
%grid
%\draw [gray] (0,0) grid (8,8);
%
\end{tikzpicture}
%=============================

%==============================
\end{urdufont}
\end{document}
%---------------------

