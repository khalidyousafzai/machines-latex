\documentclass[b5paper]{standalone}
\usepackage{fontspec}
\usepackage{polyglossia}
\usepackage{circuitikz}
\usepackage{ifthen}   
\usepackage{amsmath}
\usepackage{pgfplots}
\usetikzlibrary{calc}
%
\setmainlanguage{english}
\setotherlanguages{arabic}
\newfontfamily\arabicfont[Scale=1.0,Script=Arabic]{Scheherazade}
\newfontfamily\urdufont[Scale=1.0,Script=Arabic]{XB Tabriz}

  \newcommand\DrawControl[3]{
  node[#2,circle,fill=#2,inner sep=2pt,label={above:$#1$},label={[black]below:{\footnotesize#3}}] at #1 {}
}
\begin{document}
\begin{urdufont}


\begin{tikzpicture}[baseline,yscale=0.6]
%grid
%\draw [gray,thick] (0,0) grid (6,5);
%\draw [gray,thin,xstep=0.1,ystep=0.1] (0,0) grid (6,5);
%
\draw[] (0,0)--(6,0)node[pos=0.4,below]{\RL{برقی بار}};
\draw[] (0,0)--(0,5);
\node[rotate=90] at (-0.5,1.75){\small\RL{محرک برقی دباو}};
%\draw[help lines] (0,0) grid (8,5);
\draw[] 
  (0,0) .. controls (1,2) and  (4,3.5) .. 
  (6,4);  
\draw node[rotate=35] at (1.2,1){\RL{سلسلہ وار}};
\draw[] 
  (0,3) ..controls(2,3.5) and (4,3.5)..
  (6,3.5) ;  
\draw node[rotate=8] at (1.2,2.8){\RL{مرکب}};
\draw[] 
  (0,4) 
.. controls(2,4) and (4,3.8)..
  (6,3);  
\draw node[rotate=0] at (1.2,3.65){\RL{بیرونی}};
\draw[] 
  (0,4.5) 
.. controls (2,4.3) and (4,4.2)..
  (6,2.5);  
\draw node[rotate=-10] at (1.2,4.75){\RL{متوازی}};
%text
\draw (4.45,0)--++(0,-0.1)node[below]{$100 \%$};
\draw (0,3.5)--++(-0.1,0)node[left]{$100 \%$};
\end{tikzpicture}

\end{urdufont}
\end{document}
%---------------------

