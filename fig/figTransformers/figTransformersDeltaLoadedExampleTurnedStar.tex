\documentclass[b5paper]{standalone}
\usepackage{fontspec}
\usepackage{polyglossia}
\usepackage{circuitikz}
\usepackage{ifthen}   
\usepackage{siunitx}  
\usepackage{amsmath}
\usetikzlibrary{calc}
%
\setmainlanguage{english}
\setotherlanguages{arabic}
\newfontfamily\arabicfont[Scale=1.0,Script=Arabic]{Scheherazade}
\newfontfamily\urdufont[Scale=1.0,Script=Arabic]{XB Tabriz}

%transformer outer dimensions
\pgfmathsetmacro{\shiftX}{9 cm}
\pgfmathsetmacro{\shiftY}{6.5 cm}

\pgfmathsetmacro{\arm}{2}
\pgfmathsetmacro{\con}{2.5}
\pgfmathsetmacro{\conStarS}{\con-\arm*cos(30)}                  %one arm of star is at ninty degree
\pgfmathsetmacro{\conStarL}{\con+\arm*cos(30)}
\pgfmathsetmacro{\h}{sqrt(3)/2*\arm}
\pgfmathsetmacro{\conDeltaS}{\con-\arm/sqrt(3)}                      %one arm of delta is at 90 degree as h=sqrt{3}/2*\arm
\pgfmathsetmacro{\conDeltaL}{\con+\arm/sqrt(12)}
\pgfmathsetmacro{\conDeltaZeroS}{\con-\arm*cos(60)}    %one arm of delta is at 0 degree as h=sqrt{3}/2*\arm
\pgfmathsetmacro{\conDeltaZeroM}{\con}
\pgfmathsetmacro{\conDeltaZeroL}{\con+\arm*cos(60)}


\pgfmathsetmacro{\gap}{3 cm}        %gap between star and delta centres
\pgfmathsetmacro{\startNintyX}{-\h/3}    %start of delta if one arm is at 90 degrees
\pgfmathsetmacro{\startNintyY}{-\arm/2}
\pgfmathsetmacro{\startZeroX}{-\arm/2}   %start of delta if one arm is at 0 degrees
\pgfmathsetmacro{\startZeroY}{-\h/3} 

\begin{document}
\begin{urdufont}
\begin{tikzpicture}
%grid
%\draw[gray,thick] (-3,-3) grid (18,5);
%\draw[gray,thin,xstep=0.1,ystep=0.1] (-3,-3) grid (18,5);
%
%DELTA TRANSFORMER
\draw (\startZeroX,\startZeroY)coordinate(delA){} to [inductor] ++(0:\arm)coordinate(delB){} to [inductor] ++(120:\arm)coordinate(delC){} to [inductor] ++(-120:\arm);
\draw (delA) to [short,-o]++(180:\conDeltaZeroS);
\draw (delB) to [short] ++(0,-0.5) to [short,-o]++(180:\conDeltaZeroL);
\draw (delC) to [short,-o]++(180:\conDeltaZeroM);
%STAR TRANSFORMER
\begin{scope}[xshift=\gap]
\draw (0,0)node[below right]{$n$} to [inductor] ++(0:\arm) coordinate(starA){};
\draw (0,0) to [inductor,l={$\SI{346.41}{\volt}$}] ++(120:\arm)  to [short] ++(0,0.5) coordinate(starB){};
\draw (0,0) to [inductor] ++(-120:\arm) to [short] ++(0,-0.5)coordinate(starC){};
\node at (0,0) (starNN){};
%----------
%STAR LOAD
\begin{scope}[xshift=0.7*\shiftX]
\draw(0,0) to [resistor] ++(180:0.8*\arm) to [inductor] ++(180:0.8*\arm)coordinate(loadA){}; 
\draw(0,0) to [resistor,l^={$0.168$}] ++(60:0.8*\arm) to [inductor,l^={$j 0.0639$}] ++(60:0.8*\arm) to [short] ++(0,0.2)coordinate(loadB){}; 
\draw(0,0) to [resistor] ++(-60:0.8*\arm) to [inductor] ++(-60:0.8*\arm) to [short] ++(0,-0.2) coordinate(loadC){}; 
\draw (starA) -| (loadA);
\draw (starB) |- (loadB);
\draw (starC) |- (loadC);
\node at (0,0) (loadNN){};
\draw[gray,dashed] (starNN) --++(45:1) --++(2.5*\arm,0) --(loadNN);
\end{scope}
\end{scope}
%

\end{tikzpicture}%
\end{urdufont}
\end{document}
%---------------------

