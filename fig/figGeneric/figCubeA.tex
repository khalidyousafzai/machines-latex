\documentclass[b5paper]{standalone}
\usepackage{tikz}
\usepackage{amsmath}
\usepackage{calc}

\newcommand*{\kvec}[1]{{\ensuremath{{\boldsymbol{#1}}}}}
\newcommand*{\kvecsub}[2]{{\ensuremath{{\boldsymbol{#1}}}_{\textup{#2}}}}

\newcommand*{\ax}{\ensuremath{{\boldsymbol{a}}_{\textup{x}}}}
\newcommand*{\ay}{\ensuremath{{\boldsymbol{a}}_{\textup{y}}}}
\newcommand*{\az}{\ensuremath{{\boldsymbol{a}}_{\textup{z}}}}
%
\newcommand*{\arho}{\ensuremath{{\boldsymbol{a}}_{\rho}}}
\newcommand*{\aphi}{\ensuremath{{\boldsymbol{a}}_{\phi}}}
%
\newcommand*{\ar}{\ensuremath{{\boldsymbol{a}}_{\textup{r}}}}
\newcommand*{\atheta}{\ensuremath{{\boldsymbol{a}}_{\theta}}}

\newcommand*{\aN}{\ensuremath{{\boldsymbol{a}}_N}}

\newcommand*{\au}{\ensuremath{{\boldsymbol{a}}_u}}
\newcommand*{\av}{\ensuremath{{\boldsymbol{a}}_v}}
\newcommand*{\aw}{\ensuremath{{\boldsymbol{a}}_w}}

\newcommand*{\Ex}{\ensuremath{{\boldsymbol{E}}_x}}
\newcommand*{\Ey}{\ensuremath{{\boldsymbol{E}}_y}}
\newcommand*{\Ez}{\ensuremath{{\boldsymbol{E}}_z}}
%
\newcommand*{\Erho}{\ensuremath{{\boldsymbol{E}}_{\rho}}}
\newcommand*{\Ephi}{\ensuremath{{\boldsymbol{E}}_{\phi}}}
%
\newcommand*{\Er}{\ensuremath{{\boldsymbol{E}}_r}}
\newcommand*{\Etheta}{\ensuremath{{\boldsymbol{E}}_{\theta}}}



 \pgfmathsetmacro{\x}{2}            %start points, lower left corner
 \pgfmathsetmacro{\y}{4}
\pgfmathsetmacro{\z}{3}
%
 \pgfmathsetmacro{\xticks}{\x-1}            %number of ticks
 \pgfmathsetmacro{\yticks}{\y-1}
\pgfmathsetmacro{\zticks}{\z-1}
\pgfmathsetmacro{\tickL}{0.05}    %tick mark length

\begin{document}

\begin{tikzpicture}[x={(-0.5 cm,-0.5 cm)},y={(1 cm,0 cm)},z={(0 cm,1 cm)}]


%front of cube
\coordinate (n1) at (\x,0,0);
\coordinate (n2) at (\x,0,\z);
\coordinate (n3) at (\x,\y,\z);
\coordinate (n4) at (\x,\y,0);
%back of cube
\coordinate (n5) at (0,0,0);
\coordinate (n6) at (0,0,\z);
\coordinate (n7) at (0,\y,\z);
\coordinate (n8) at (0,\y,0);
%axis
\draw[gray] (0,0,0)--(\x+0.5,0,0);
\draw[gray] (0,0,0)--(0,\y+1,0);
\draw[gray] (0,0,0)--(0,0,\z+0.5);
%x,y,z length labels
\draw[left] node at (\x,0,0){$x$};
\draw[below] node at (0,\y,0){$y$};
\draw[left] node at (0,0,\z){$z$};
%ticks
\foreach \xt in {1,...,\xticks}{\draw[gray] (\xt,-\tickL,0)--(\xt,\tickL,0);}
\foreach \yt in {1,...,\yticks}{\draw[gray] (-\tickL,\yt,0)--(\tickL,\yt,0);}
\foreach \zt in {1,...,\zticks}{\draw[gray] (0,-\tickL,\zt)--(0,\tickL,\zt);}
%cube
\draw[gray,thin] (n1)--(n2)--(n3)--(n4)--cycle;   %front side
\draw[gray,thin] (n2)--(n6)--(n7)--(n3)--cycle;   %top side
\draw[gray,thin] (n3)--(n7)--(n8)--(n4)--cycle;     %right side
%vectors
\draw[-latex] (0,0,0)--(\x,\y,\z)node[pos=0.5,above left]{${\bf{A}}$} node[above]{$P$};
\draw[-latex] (0,0,0)--(\x,0,0)node[pos=0.3,left]{${\bf{A}}_x$};
\draw[-latex] (\x,0,0)--(\x,\y,0)node[pos=0.5,below]{${\bf{A}}_y$};
\draw[-latex] (\x,\y,0)--(\x,\y,\z)node[pos=0.5,right]{${\bf{A}}_z$};
%TEXT
\draw node at (0,\y+2,\z){$\begin{aligned} &P(x,y,z)\\ & \kvec{A}=\kvec{A}_x +\kvec{A}_y+\kvec{A}_z\\   &\kvec{A}=x \ax+y \ay+z\az \end{aligned}$};
%lines on body
%\foreach \delX in {0,...,\x}{
%\draw[gray,thin](\delX,\y,0)--(\delX,\y,\z);
%\draw[gray,thin](\delX,0,\z)--(\delX,\y,\z);   }
%
%\foreach \delY in {0,...,\y}{
%\draw[gray,thin](\x,\delY,0)--(\x,\delY,\z);
%\draw[gray,thin](0,\delY,\z)--(\x,\delY,\z);  }
%
%\foreach \delZ  in {0,...,\z}{
%\draw[gray,thin](0,\y,\delZ)--(\x,\y,\delZ);
%\draw[gray,thin](\x,0,\delZ)--(\x,\y,\delZ);  }
%
%
%\shade[left color=transparent!0, right color=transparent!100,shading angle=105,fill opacity=0.5,draw=gray] (n1)--(n2)--(n3)--(n4)--cycle;   %front side
%\shade[right color=gray!10,left color=black!50,shading angle=105,fill opacity=0.5,draw=gray] (n2)--(n6)--(n7)--(n3)--cycle;   %top side
%\shade[right color=gray!10,left color=black!50,shading angle=105,fill opacity=0.5,draw=gray] (n3)--(n7)--(n8)--(n4)--cycle;     %right side
%cube repeated a bit differently
%\shade[left color=transparent!100, right color=transparent!0,shading angle=105,fill opacity=0.5] (n1)--(n2)--(n3)--(n4)--cycle;   %front side
%\shade[left color=transparent!100, right color=transparent!0,shading angle=45,fill opacity=0.5] (n2)--(n6)--(n7)--(n3)--cycle;   %top side
%\shade[left color=transparent!0, right color=transparent!100,shading angle=105,fill opacity=0.5] (n3)--(n7)--(n8)--(n4)--cycle;     %right side
%

\end{tikzpicture}

\end{document}
